\chapter{lcd.c}
%\begin{verbatimtab}
\begin{lstlisting}[language=C, caption={lcd.c}, label=lcd-c, captionpos=b]
/*************************************************
LCD
Author: Sergio Santos 
<sergio.salazar.santos@gmail.com>
License: GNU General Public License
Hardware: all
Date: 25102020
Comment:
Tested Atemga128 16Mhz and Atmega328 8Mhz                    
*************************************************/
#ifndef F_CPU
/***Mandatory to use util/delay.h***/
#define F_CPU 16000000UL
#endif
/***Library***/
#include <avr/io.h>
#include <util/delay.h>
#include <inttypes.h>
#include "lcd.h"
/***Constant & Macro***/
#ifndef GLOBAL_INTERRUPT_ENABLE
#define GLOBAL_INTERRUPT_ENABLE 7
#endif
//CMD RS
#define INST 0
#define DATA 1
//ticks depends on CPU frequency this case 16Mhz
#define LCD_N_TICKS 3
#define LCD_BF_TICKS 10
/***Global File Variable***/
volatile uint8_t *lcd0_DDR;
volatile uint8_t *lcd0_PIN;
volatile uint8_t *lcd0_PORT;
uint8_t lcd0_detect;
volatile uint8_t *lcd1_DDR;
volatile uint8_t *lcd1_PIN;
volatile uint8_t *lcd1_PORT;
uint8_t lcd1_detect;
/***Header***/
void LCD0_inic(void);
void LCD0_write(char c, unsigned short D_I);
char LCD0_read(unsigned short D_I);
void LCD0_BF(void);
void LCD0_putch(char c);
char LCD0_getch(void);
void LCD0_string(const char* s); // RAW
void LCD0_string_size(const char* s, uint8_t size); // RAW
void LCD0_hspace(uint8_t n);
void LCD0_clear(void);
void LCD0_gotoxy(unsigned int y, unsigned int x);
void LCD0_strobe(unsigned int num);
void LCD0_reboot(void);
void LCD1_inic(void);
void LCD1_write(char c, unsigned short D_I);
char LCD1_read(unsigned short D_I);
void LCD1_BF(void);
void LCD1_putch(char c);
char LCD1_getch(void);
void LCD1_string(const char* s);
void LCD1_string_size(const char* s, uint8_t size); // RAW
void LCD1_hspace(uint8_t n);
void LCD1_clear(void);
void LCD1_gotoxy(unsigned int y, unsigned int x);
void LCD1_strobe(unsigned int num);
void LCD1_reboot(void);
unsigned int LCD_ticks(unsigned int num);
/***Procedure & Function***/
LCD0 LCD0enable(volatile uint8_t *ddr, volatile uint8_t *pin, volatile uint8_t *port)
{
	//LOCAL VARIABLES
	uint8_t tSREG;
	tSREG=SREG;
	SREG&=~(1<<GLOBAL_INTERRUPT_ENABLE);
	//ALLOCACAO MEMORIA PARA Estrutura
	LCD0 lcd0;
	//import parametros
	lcd0_DDR=ddr;
	lcd0_PIN=pin;
	lcd0_PORT=port;
	//inic variables
	*lcd0_DDR=0x00;
	*lcd0_PORT=0xFF;
	lcd0_detect=*lcd0_PIN & (1<<NC);
	//Direccionar apontadores para PROTOTIPOS
	lcd0.write=LCD0_write;
	lcd0.read=LCD0_read;
	lcd0.BF=LCD0_BF;
	lcd0.putch=LCD0_putch;
	lcd0.getch=LCD0_getch;
	lcd0.string=LCD0_string;
	lcd0.string_size=LCD0_string_size; // RAW
	lcd0.hspace=LCD0_hspace;
	lcd0.clear=LCD0_clear;
	lcd0.gotoxy=LCD0_gotoxy;
	lcd0.reboot=LCD0_reboot;
	//LCD INIC
	LCD0_inic();
	SREG=tSREG;
	//
	return lcd0;
}
void LCD0_inic(void)
{
	//LCD INIC
	*lcd0_DDR=(1<<RS)|(1<<RW)|(1<<EN)|(0<<NC);
	*lcd0_PORT=(1<<NC);
	/***INICIALIZACAO LCD**datasheet*/
	_delay_ms(40);
	LCD0_write(0x33,INST); //function set
	_delay_us(39);
	LCD0_write(0x33,INST); //function set
	_delay_us(39);
	LCD0_write(0x2B,INST); //function set
	_delay_us(37);
	LCD0_write(0x0C,INST);// display on/off control
	_delay_us(37);
	LCD0_write(0x01,INST);// clear display
	_delay_ms(1.53);
	LCD0_write(0x06,INST);// entry mode set (crazy settings)
	_delay_us(37);
	/***INICIALIZATION END***/
	//LCD0_write(0x1F,INST);// cursor or display shift
	//_delay_us(39);
	//LCD0_write(0x03,INST);// return home
	//_delay_ms(1.53);
}
void LCD0_write(char c, unsigned short D_I)
{
	*lcd0_PORT&=~(1<<RW);//lcd as input WRITE INSTRUCTION
	if(D_I) *lcd0_PORT|=(1<<RS); else *lcd0_PORT&=~(1<<RS);
	*lcd0_DDR|=(1<<DB4)|(1<<DB5)|(1<<DB6)|(1<<DB7);//mcu as output
	*lcd0_PORT|=(1<<EN);
	if(c & 0x80) *lcd0_PORT|=1<<DB7; else *lcd0_PORT&=~(1<<DB7);
	if(c & 0x40) *lcd0_PORT|=1<<DB6; else *lcd0_PORT&=~(1<<DB6);
	if(c & 0x20) *lcd0_PORT|=1<<DB5; else *lcd0_PORT&=~(1<<DB5);
	if(c & 0x10) *lcd0_PORT|=1<<DB4; else *lcd0_PORT&=~(1<<DB4);
	*lcd0_PORT&=~(1<<EN);
	LCD_ticks(LCD_N_TICKS);
	*lcd0_PORT|=(1<<EN);
	if(c & 0x08) *lcd0_PORT|=1<<DB7; else *lcd0_PORT&=~(1<<DB7);
	if(c & 0x04) *lcd0_PORT|=1<<DB6; else *lcd0_PORT&=~(1<<DB6);
	if(c & 0x02) *lcd0_PORT|=1<<DB5; else *lcd0_PORT&=~(1<<DB5);
	if(c & 0x01) *lcd0_PORT|=1<<DB4; else *lcd0_PORT&=~(1<<DB4);
	*lcd0_PORT&=~(1<<EN);
	LCD_ticks(LCD_N_TICKS);
}
char LCD0_read(unsigned short D_I)
{
	char c=0x00;
	*lcd0_DDR&=~((1<<DB4)|(1<<DB5)|(1<<DB6)|(1<<DB7));//mcu as input
	*lcd0_PORT|=(1<<DB4)|(1<<DB5)|(1<<DB6)|(1<<DB7);//pullup resistors
	*lcd0_PORT|=(1<<RW);//lcd as output READ INSTRUCTION
	if(D_I) *lcd0_PORT|=(1<<RS); else *lcd0_PORT&=~(1<<RS);
	*lcd0_PORT|=(1<<EN);
	if(*lcd0_PIN & (1<<DB7)) c|=1<<7; else c&=~(1<<7);
	if(*lcd0_PIN & (1<<DB6)) c|=1<<6; else c&=~(1<<6);
	if(*lcd0_PIN & (1<<DB5)) c|=1<<5; else c&=~(1<<5);
	if(*lcd0_PIN & (1<<DB4)) c|=1<<4; else c&=~(1<<4);
	*lcd0_PORT&=~(1<<EN);
	LCD_ticks(LCD_N_TICKS);
	*lcd0_PORT|=(1<<EN);
	if(*lcd0_PIN & (1<<DB7)) c|=1<<3; else c&=~(1<<3);
	if(*lcd0_PIN & (1<<DB6)) c|=1<<2; else c&=~(1<<2);
	if(*lcd0_PIN & (1<<DB5)) c|=1<<1; else c&=~(1<<1);
	if(*lcd0_PIN & (1<<DB4)) c|=1<<0; else c&=~(1<<0);
	*lcd0_PORT&=~(1<<EN);
	LCD_ticks(LCD_N_TICKS);
	return c;
}
void LCD0_BF(void)
{
	unsigned int i;
	char inst=0x80;
	for(i=0;0x80&inst;i++){
		inst=LCD0_read(INST);
		LCD_ticks(LCD_BF_TICKS);
		if(i>10)// if something goes wrong
		break;
	}
}
void LCD0_putch(char c)
{
	LCD0_write(c,DATA);
	LCD0_BF();
}
char LCD0_getch(void)
{
	char c;
	c=LCD0_read(DATA);
	LCD0_BF();
	return c;
}
void LCD0_string(const char* s)
{
	char tmp;
	while(*s){
		tmp=*(s++);
		LCD0_write(tmp,DATA);
		LCD0_BF();
	}
}
void LCD0_string_size(const char* s, uint8_t size)
{
	char tmp;
	uint8_t pos=0;
	while(*s){
		tmp=*(s++);
		pos++;
		if(pos>size) // 1 TO SIZE+1
		break;
		LCD0_write(tmp,DATA);
		LCD0_BF();
	}
	while(pos<size){ // TO SIZE
		pos++;
		LCD0_write(' ',DATA);
		LCD0_BF();
	}
}
void LCD0_hspace(uint8_t n)
{
	for(;n;n--){
		LCD0_write(' ',DATA);
		LCD0_BF();
	}
}
void LCD0_clear(void)
{
	LCD0_write(0x01,INST);
	_delay_ms(1.53);
}
void LCD0_gotoxy(unsigned int y, unsigned int x)
{
	switch(y){
		case 0:
		LCD0_write((0x80+x),INST);
		LCD0_BF();
		break;
		case 1:
		LCD0_write((0xC0+x),INST);
		LCD0_BF();
		break;
		case 2:
		LCD0_write((0x94+x),INST);//0x94
		LCD0_BF();
		break;
		case 3:
		LCD0_write((0xD4+x),INST);//0xD4
		LCD0_BF();
		break;
		default:
		break;
	}
}
void LCD0_strobe(unsigned int num)
{
	*lcd0_PORT|=(1<<EN);
	LCD_ticks(num);
	*lcd0_PORT&=~(1<<EN);
}
void LCD0_reboot(void)
{
	uint8_t tSREG;
	tSREG=SREG;
	SREG&=~(1<<GLOBAL_INTERRUPT_ENABLE);
	//low high detect pin NC
	uint8_t i;
	uint8_t tmp;
	tmp=*lcd0_PIN & (1<<NC);
	i=tmp^lcd0_detect;
	i&=tmp;
	if(i)
	LCD0_inic();
	lcd0_detect=tmp;
	SREG=tSREG;
}
/************************************************/
LCD1 LCD1enable(volatile uint8_t *ddr, volatile uint8_t *pin, volatile uint8_t *port)
{
	//LOCAL VARIABLES
	uint8_t tSREG;
	tSREG=SREG;
	SREG&=~(1<<GLOBAL_INTERRUPT_ENABLE);
	//ALLOCACAO MEMORIA PARA Estrutura
	LCD1 lcd1;
	//import parametros
	lcd1_DDR=ddr;
	lcd1_PIN=pin;
	lcd1_PORT=port;
	//inic variables
	*lcd1_DDR=0x00;
	*lcd1_PORT=0xFF;
	lcd1_detect=*lcd1_PIN & (1<<NC);
	//Direccionar apontadores para PROTOTIPOS
	lcd1.write=LCD1_write;
	lcd1.read=LCD1_read;
	lcd1.BF=LCD1_BF;
	lcd1.putch=LCD1_putch;
	lcd1.getch=LCD1_getch;
	lcd1.string=LCD1_string;
	lcd1.string_size=LCD1_string_size; // RAW
	lcd1.hspace=LCD1_hspace;
	lcd1.clear=LCD1_clear;
	lcd1.gotoxy=LCD1_gotoxy;
	lcd1.reboot=LCD1_reboot;
	//LCD INIC
	LCD1_inic();
	SREG=tSREG;
	//
	return lcd1;
}
void LCD1_inic(void)
{
	//LCD INIC
	*lcd1_DDR=(1<<RS)|(1<<RW)|(1<<EN)|(0<<NC);
	*lcd1_PORT=(1<<NC);
	/***INICIALIZACAO**LCD**datasheet***/
	_delay_ms(40);
	LCD1_write(0x33,INST); //function set
	_delay_us(39);
	LCD1_write(0x33,INST); //function set
	_delay_us(39);
	LCD1_write(0x2B,INST); //function set
	_delay_us(37);
	LCD1_write(0x0C,INST);// display on/off control
	_delay_us(37);
	LCD1_write(0x01,INST);// clear display
	_delay_ms(1.53);
	LCD1_write(0x06,INST);// entry mode set (crazy settings)
	_delay_us(37);
	/***INICIALIZATION END***/
	//LCD1_write(0x1F,INST);// cursor or display shift
	//_delay_us(39);
	//LCD1_write(0x03,INST);// return home
	//_delay_ms(1.53);
}
void LCD1_write(char c, unsigned short D_I)
{
	*lcd1_PORT&=~(1<<RW);//lcd as input WRITE INSTRUCTION
	if(D_I) *lcd1_PORT|=(1<<RS); else *lcd1_PORT&=~(1<<D_I);
	*lcd1_DDR|=(1<<DB4)|(1<<DB5)|(1<<DB6)|(1<<DB7);//mcu as output
	*lcd1_PORT|=(1<<EN);
	if(c & 0x80) *lcd1_PORT|=1<<DB7; else *lcd1_PORT&=~(1<<DB7);
	if(c & 0x40) *lcd1_PORT|=1<<DB6; else *lcd1_PORT&=~(1<<DB6);
	if(c & 0x20) *lcd1_PORT|=1<<DB5; else *lcd1_PORT&=~(1<<DB5);
	if(c & 0x10) *lcd1_PORT|=1<<DB4; else *lcd1_PORT&=~(1<<DB4);
	*lcd1_PORT&=~(1<<EN);
	LCD_ticks(LCD_N_TICKS);
	*lcd1_PORT|=(1<<EN);
	if(c & 0x08) *lcd1_PORT|=1<<DB7; else *lcd1_PORT&=~(1<<DB7);
	if(c & 0x04) *lcd1_PORT|=1<<DB6; else *lcd1_PORT&=~(1<<DB6);
	if(c & 0x02) *lcd1_PORT|=1<<DB5; else *lcd1_PORT&=~(1<<DB5);
	if(c & 0x01) *lcd1_PORT|=1<<DB4; else *lcd1_PORT&=~(1<<DB4);
	*lcd1_PORT&=~(1<<EN);
	LCD_ticks(LCD_N_TICKS);
}
char LCD1_read(unsigned short D_I)
{
	char c=0x00;
	*lcd1_DDR&=~((1<<DB4)|(1<<DB5)|(1<<DB6)|(1<<DB7));//mcu as input
	*lcd1_PORT|=(1<<DB4)|(1<<DB5)|(1<<DB6)|(1<<DB7);//pullup resistors
	*lcd1_PORT|=(1<<RW);//lcd as output READ INSTRUCTION
	if(D_I) *lcd1_PORT|=(1<<RS); else *lcd1_PORT&=~(1<<D_I);
	*lcd1_PORT|=(1<<EN);
	if(*lcd1_PIN & (1<<DB7)) c|=1<<7; else c&=~(1<<7);
	if(*lcd1_PIN & (1<<DB6)) c|=1<<6; else c&=~(1<<6);
	if(*lcd1_PIN & (1<<DB5)) c|=1<<5; else c&=~(1<<5);
	if(*lcd1_PIN & (1<<DB4)) c|=1<<4; else c&=~(1<<4);
	*lcd1_PORT&=~(1<<EN);
	LCD_ticks(LCD_N_TICKS);
	*lcd1_PORT|=(1<<EN);
	if(*lcd1_PIN & (1<<DB7)) c|=1<<3; else c&=~(1<<3);
	if(*lcd1_PIN & (1<<DB6)) c|=1<<2; else c&=~(1<<2);
	if(*lcd1_PIN & (1<<DB5)) c|=1<<1; else c&=~(1<<1);
	if(*lcd1_PIN & (1<<DB4)) c|=1<<0; else c&=~(1<<0);
	*lcd1_PORT&=~(1<<EN);
	LCD_ticks(LCD_N_TICKS);
	return c;
}
void LCD1_BF(void)
{
	unsigned int i;
	char inst=0x80;
	for(i=0;0x80&inst;i++){
		inst=LCD1_read(INST);
		LCD_ticks(LCD_BF_TICKS);
		if(i>10)// if something goes wrong
		break;
	}
}
void LCD1_putch(char c)
{
	LCD1_write(c,DATA);
	LCD1_BF();
}
char LCD1_getch(void)
{
	char c;
	c=LCD1_read(DATA);
	LCD1_BF();
	return c;
}
void LCD1_string(const char* s)
{
	char tmp;
	while(*s){
		tmp=*(s++);
		LCD1_write(tmp,DATA);
		LCD1_BF();
	}
}
void LCD1_string_size(const char* s, uint8_t size)
{
	char tmp;
	uint8_t pos=0;
	while(*s){
		tmp=*(s++);
		pos++;
		if(pos>size)
		break;
		LCD1_write(tmp,DATA);
		LCD1_BF();
	}
	while(pos<size){
		pos++;
		LCD1_write(' ',DATA);
		LCD1_BF();
	}
}
void LCD1_hspace(uint8_t n)
{
	for(;n;n--){
		LCD1_write(' ',DATA);
		LCD1_BF();
	}
}
void LCD1_clear(void)
{
	LCD1_write(0x01,INST);
	_delay_ms(1.53);
}
void LCD1_gotoxy(unsigned int y, unsigned int x)
{
	switch(y){
		case 0:
		LCD1_write((0x80+x),INST);
		LCD1_BF();
		break;
		case 1:
		LCD1_write((0xC0+x),INST);
		LCD1_BF();
		break;
		case 2:
		LCD1_write((0x94+x),INST);//0x94
		LCD1_BF();
		break;
		case 3:
		LCD1_write((0xD4+x),INST);//0xD4
		LCD1_BF();
		break;
		default:
		break;
	}
}
void LCD1_strobe(unsigned int num)
{
	*lcd1_PORT|=(1<<EN);
	LCD_ticks(num);
	*lcd1_PORT&=~(1<<EN);
}
void LCD1_reboot(void)
{
	//LOCAL VARIABLES
	uint8_t tSREG;
	tSREG=SREG;
	SREG&=~(1<<GLOBAL_INTERRUPT_ENABLE);
	//low high detect pin NC
	uint8_t i;
	uint8_t tmp;
	tmp=*lcd1_PIN & (1<<NC);
	i=tmp^lcd1_detect;
	i&=tmp;
	if(i)
	LCD1_inic();
	lcd1_detect=tmp;
	SREG=tSREG;
}
unsigned int LCD_ticks(unsigned int num)
{
	unsigned int count;
	for(count=0;count<num;count++)
	;
	return count;
}
/***Interrupt***/
/***Comment***
*************/
/***EOF***/
\end{lstlisting}
%\end{verbatimtab}
%%%%%%%%%%%%%%%%%%%%%%%%%%%%%%%%%%%%%%%%%%%%%%%%%%%%%%%%%%
