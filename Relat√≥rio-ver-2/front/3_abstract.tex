%---------------------------------------------------------
%	ABSTRACT PAGES
%---------------------------------------------------------
% IMPORTANT NOTE: the abstract must always be written in two languages. If the report
% is written in Portuguese you have selected 'portuguese' as the language in the document class.
% Therefore, the portuguese version of the abstract must come first, so write it in the
% below area denoted by 'MAIN LANGUAGE ABSTRACT'. The english version follows in the
% 'SECOND LANGUAGE ABSTRACT' section.
% If the report is written in English, first will come the abstract in English
% ('MAIN LANGUAGE ABSTRACT') and then in Portuguese ('SECOND LANGUAGE ABSTRACT').
\begin{abstract}
%%%%%%%%%%%%%%%% MAIN LANGUAGE ABSTRACT %%%%%%%%%%%%%%%%%%
%É a apresentação do trabalho, onde se pretende que seja feito um resumo do mesmo
\par O projeto proposto prevê a implementação de uma balança digital, utilizando um microcontrolador. Vai ser utilizado uma célula de carga como sensor de conversão do peso a medir para uma diferença de potencial proporcional em tensão (\textit{Volt}).
\\
\\
Após a aquisição deste sinal em tensão gerado pela ponte \textit{Wheatstone} da célula de carga, será depois ligado a um amplificador de sinal com \ac{adc} dedicado para este tipo de aplicação, este tem 24 \textit{bits} de resolução, amplificação programável e uma taxa de transferência fisicamente configurável. O dito amplificador de célula de carga é designado por \textbf{HX711}, e possui um protocolo de comunicação dedicado. A comunicação série será realizada através do uso de um \ac{mcu}.
\\
\\
A programação do \acrshort{mcu}, o código das bibliotecas e ou drivers vai ser implementada em linguagem \textbf{C}. O objetivo é implementar uma balança funcional de fácil utilização e calibração, economicamente viável, tornado-a desta forma numa balança acessível e prática.
\\
\\
A balança digital vai ter um \textit{display} \ac{lcd} para visualizar o valor do massa medida e o parâmetro de calibração da célula de carga. Vai ter um botão de \textit{offset}, e botões para carregar os parâmetros de \textit{default} e para alterar o valor de calibração, também vai utilizar \textit{leds} de indicação para mostrar em que estado a balança se encontra.
%Este trabalho é orientado para quem tem alguns conhecimentos básicos de eletrotecnia, com orientação para a eletrónica e programação.
\\
\\
%---------------------------------------------------------
\vfill
%\vspace*{10mm} 
\noindent
\textbf{\keywordslabel}: \textit{Strain Gauge}, \textit{Load Cell}, Amplificador, Código, Programação, \textit{Embedded System}.
%%%%%%%%%%% END OF THE MAIN LANGUAGE ABSTRACT %%%%%%%%%%%%
\end{abstract}
\begin{secondlangabstract}
%%%%%%%%%%%%%%%% SECOND LANGUAGE ABSTRACT %%%%%%%%%%%%%%%%
\par The proposed project is to build a weighing scale using a microcontroller, an embedded system.
\\
\\
A load cell is what is going to be used to convert the measured weight into a potential difference by its Wheatstone configuration, generating a proportional signal.
\\
\\
After obtaining this signal it is going to be connected to a load cell amplifier with a \ac{adc} dedicated for this type of application, it has a 24 bit resolution, programable amplification and sampling rate physically programmable, this is the \textbf{HX711} chip, having a proprietary communication protocol.
This information then will be passed to the microcontroller.
\\
\\
The \ac{mcu} programming code, libraries and or drivers are to be implemented using the \textbf{C} language. The objective of this project is to get a functional weighing scale easy to use and to calibrate economically viable, in order to have a practical weight scale.
\\
\\
This weight scale will have integrated an \ac{lcd} display with buttons and led indicators, making it easy for the user to work with.
%This work is targeted to a public with some basic electrical knowledge with electronic and programming background.
%---------------------------------------------------------
\vfill
%\vspace*{10mm} 
\noindent
\textbf{\keywordslabel}: Strain Gauge, Load Cell, Amplifier, Code, Programming, Embedded System.
%%%%%%%%%% END OF THE SECOND LANGUAGE ABSTRACT %%%%%%%%%%%
\end{secondlangabstract}
%%%%%%%%%%%%%%%%%%%%%%%%%%%%%%%%%%%%%%%%%%%%%%%%%%%%%%%%%%
