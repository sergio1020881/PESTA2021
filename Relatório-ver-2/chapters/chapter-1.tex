\chapter{Introdução}
Num mundo altamente mecanizado, a força e o torque dentro de todas as grandezas, são as mais comuns. Estas têm um papel significativo nos aparelhos de medição de massa e células de carga usadas na indústria e retalho, nos automóveis e nas aeronaves, no aperto de tampas de frascos de medicina e parafusos.\cite{book-9}
\\
\\
A massa é uma das grandezas fundamentais, uma propriedade intrínseca de um objeto, que se mede pela sua resistência à aceleração.\cite{book-2}
%melhorar imagen
\begin{figure}[H]
	\centering
	\includegraphics[width=.8\linewidth]{./image/PESTA/fisica/mass-1.png}
	\caption{Peso e Massa}
	\label{mass}
\end{figure}
Através da medição do peso, isto é, da força gravitacional exercida num dado objeto, podemos calcular a massa. \cite{book-2}
\\
\\
A massa dos objetos pode ser medida por balanças utilizando vários métodos, tais como as alavancas convencionais, ou por  molas, ou ainda sistemas mais modernos, como, as balanças digitais.
\\
\\
As balanças digitais estão presentes no nosso dia a dia e tornaram-se numa ferramenta indispensável na indústria, comércio e laboratórios. Determinar a massa dos objetos facilitou a moeda de troca no comércio, e tornou-se num pilar fundamental da física e seu desenvolvimentos.
\\
\\
Existe uma procura, tanto na área comercial como na industrial, que empurra o desenvolvimento dos sensores para obter melhores resultados, quanto à precisão e imunidade de influências exteriores na medição das grandezas mencionadas (força, torque e a massa).
\\
\\
A importância desta grandeza (massa) foi o mote para o projeto proposto: a implementação de uma balança digital para uso doméstico, recorrendo a componentes que estão disponíveis no mercado.
%%%%%%%%%%%%%%%%%%%%%%%%%%%%%%%%%%%%%%%%%%%%%%%%%%%%%%%%%%
\section{Objetivos}
O objetivo principal deste projeto será a implementação de uma balança digital economicamente viável, usando os sensores e equipamentos disponíveis no mercado, para ter um produto útil e fácil de ser replicado.
\\
\\
Escolher o sensor adequado e meios de tratamento da informação e comunicação, será o objeto de estudo. Tendo como foco a utilização de um \textit{Embedded System} utilizando as ferramentas necessárias para o executar.
\\
\\
No final, o produto obtido será testado, de forma a tentar aperfeiçoar, fazer melhorias, alterações e adaptações que possam surgir, tendo em vista a possibilidade de simplificar o projeto e torná-lo mais atraente ao consumidor.
%%%%%%%%%%%%%%%%%%%%%%%%%%%%%%%%%%%%%%%%%%%%%%%%%%%%%%%%%%
\section{Calendarização}
O segundo semestre teve início em 8 de Março, num ambiente de pandemia COVID-19, período durante o qual foi necessário recorrer ao ensino à distância, e todo o trabalho teve de ser acompanhado \textit{online} pelos docentes. No entanto foi necessário organizar as tarefas pretendidas de acordo com o plano abaixo descrito, para poder fazer a entrega deste trabalho antes da data limite da época normal (28 de Junho de 2021).
\begin{table}[H]
	\caption{Calendarização das tarefas}
	%\begin{sidewaysfigure}
	\begin{ganttchart}[vgrid, hgrid]{1}{20}
		\gantttitle{Março}{5} 
		\gantttitle{Abril}{5}
		\gantttitle{Maio}{5}
		\gantttitle{Juno}{5}\\
		\gantttitlelist{1,...,20}{1}\\
		%First Group
		\ganttgroup{Requisitos}{2}{10} \\
		\ganttbar{Material}{3}{5} \\
		\ganttbar{\textit{Template} LaTeX}{5}{10} \\
		\ganttbar{\textbf{IDE} \textit{Template}}{3}{10}\\
		%\ganttlink{elem0}{elem1}
		%\ganttlink{elem1}{elem2}
		%\ganttlink{elem2}{elem3}
		%\ganttmilestone{Milestone 1}{11}
		%Second Group
		\ganttgroup{Projecto}{3}{20} \\
		\ganttbar{Kit Desenvolvimento}{3}{4} \\
		\ganttbar{Montagen Mesa Sensor}{5}{7} \\%5 7
		\ganttbar{HX711 comunicação}{4}{10} \\%4 8
		\ganttbar{Programação e ensaio}{4}{20}\\
		%\ganttlink{elem4}{elem5}
		%\ganttlink{elem5}{elem6}
		%\ganttlink{elem6}{elem7}
		%\ganttmilestone{Milestone 1}{11}
		%Third Group
		\ganttgroup{Relatório}{9}{20} \\
		\ganttbar{Literatura}{8}{12} \\
		\ganttbar{Análise documentação}{9}{12} \\
		%\ganttbar{Validação}{10}{12} \\
		\ganttbar{Execução}{10}{20}
		%\ganttlink{elem8}{elem9}
		%\ganttlink{elem9}{elem10}
		%\ganttlink{elem10}{elem11}
		%\ganttmilestone{Milestone 1}{11}
	\end{ganttchart}
	\label{gantt}
%\end{sidewaysfigure}
\end{table}
%%%%%%%%%%%%%%%%%%%%%%%%%%%%%%%%%%%%%%%%%%%%%%%%%%%%%%%%%%
\section{Organização do Relatório}
No capítulo 1 é feita uma contextualização do trabalho realizado e do seu propósito face às necessidades do mundo atual.
\\
\\
No capítulo 2 é apresentada uma breve história da evolução das balanças, e depois um estudo da balança eletrónica.
\\
\\
No capitulo 3 é apresentado a balança digital, seus componentes e características com uma descrição.
\\
\\
No capitulo 4 é descrito o software utilizado, estrutura do programa, e a implementação do código.
\\
\\
O capitulo 5 é apresentada a validação da implementação do projeto e seu funcionamento.
\\
\\
O capitulo 6 são expostas as conclusões e possíveis alterações do produto.
%%%%%%%%%%%%%%%%%%%%%%%%%%%%%%%%%%%%%%%%%%%%%%%%%%%%%%%%%%
