%%%%%%%%%%%%%%%%%%%%%%%%%%%%POSTER%%%%%%%%%%%%%%%%%%%%%%%%%%%%%%
\block{Introdução}{
	As balanças foram criadas por necessidade durante o desenvolvimento de comercio na antiguidade, os produtos que não recorriam a contagem por unidades, tais como objetos irregulares por exemplo o ouro tinham de se quantificar seu valor, e a forma de medir sua massa tornou-se numa variável de medição para troca de bens. As primeiras balanças eram alavancas em equilíbrio [ F1 × b1c = F2 × b2c ], onde nos extremos eram colocados cestos e se colocava os pesos, este estava centrado no seu centro de massa, assim se os pesos nos dois cestos serem iguais fica em equilíbrio, era um sistema de comparar com pesos fixos estabelecidos como norma (contra-pesos).\\
	Os métodos de medir a massa de objetos não conheceu nenhumas melhorias tecnológicas relevantes até a era industrial. Só nos anos do século XVIII é que o meio de medir a massa de objetos não dependia de contra-pesos. As balanças por molas foi inventado por Richard Salter, um fabricante de balanças por volta dos anos de 1770 na Inglaterra.\\
	O que vai ser utilizado no projeto vai ser um \textbf{célula de peso}, estas células tem quatro sensores \textit{strain gauges} ligadas em ponte wheatstone que vão detetar a distorção (pressão) do material, ou seja, a célula de peso e gerar um sinal em tensão proporcional a força exercida quando alimentado. Seque o mesmo principio de uma mola [ $K= \frac{\Delta l}{F}$ ].
}
\begin{columns}
\column{0.1}
\column{0.8}
\block{Kit Desenvolvimento}{
	\begin{tikzfigure}
		\includegraphics[width=0.7\textwidth]{./image/PESTA/kit/Kit_Desenvolvimento_2.jpg}
	\end{tikzfigure}
}
\end{columns}
\begin{columns}
\column{0.5}
\block{Sensor}{
	O sensor é um transdutor utilizado em converter energia de uma natureza para outra como entrada de sinal para um sistema, são os elementos principais de interface com o mundo real para o analógico. \\
	Neste projecto é o caso de utilizar uma celula de peso, ou seja, sensores piezoresistivos para determinar a massa dos objectos.
	
}
\column{0.5}
\block{Informação e Controlo}{
	Tratamento dos valores medidos, conversão de sinal medido pelo sensor para analógico, conversão de analógico para digital usando medidas discretas, tirando varias amostras para obter uma média depois dedução de sua massa.\\
	Um \textit{Frontend} de controlo intuitivo, com respectivo \textit{offset} e meio de calibração para a celula que estiver a ser usada. \qquad Uma balança electrónica simples.
}
\end{columns}
\begin{columns}
\column{1}
\block{Conclusões}{
	\bigskip
	\coloredbox{
		\hspace{.2cm}
		\textbf{Importancia:}\\ \\
		\begin{minipage}[b!]{.37\linewidth}
			\begin{itemize}
				\item Acumolação de conhecimentos e documentação (\textbf{\textit{github}}).
				\item Utilização das ferramentas (multimetro, osciloscopio, \textbf{IDE}, etc).
				\item Leitura e interpretação de \textit{datasheets} e manuais.
			\end{itemize}
		\end{minipage}
		\begin{minipage}[b!]{.3\linewidth}
			\begin{itemize}
				\item Criar uma methologia de trabalho estavel.
				\item Pesquisar de literatura.
				\item Dominar a linguagem \textbf{C}.
			\end{itemize}
		\end{minipage}
		\begin{minipage}[b!]{.4\linewidth}
			\begin{itemize}
				\item Criar drivers e livrarias independentes.
				\item Obter \textit{Know How} em diversas areas.
				\item Experimentar.
			\end{itemize}
		\end{minipage}
	}
	\bigskip
	\bigskip
	\coloredbox{
		\centering
		\textbf{Efeitos Observados:} \\
		\begin{minipage}[b!]{\linewidth}
			\hspace{3cm}
			\begin{minipage}[b!]{.2\linewidth}
			\begin{itemize}
				\item Ruido
				\item Temperatura
				\item Humidade
				\item \textit{Long term shift}
				\item Histerese
			\end{itemize}
			\end{minipage}
			\begin{minipage}[b!]{.6\linewidth}
				Ao executar este projecto teve de se ter em conta interferencia exteriores, tais como colocar o amplificador de sinal o mais proximo do sensor e afastado do ruido do circuito electrónico. Foi observado os efeito a longo prazo das leituras e seu desvio com fenomeno de histerese.
				\vspace{1cm}
			\end{minipage}
		\end{minipage}
	}
	\vspace{5cm}
}
\note[targetoffsetx=-34cm, targetoffsety=-8.5cm, width=.3\linewidth]{\bigskip
	\texttt{https://github.com/sergio1020881/PESTA2021}
}
\end{columns}
%%%%%%%%%%%%%%%%%%%%%%%%%%%%%%%%%%%%%%%%%%%%%%%%%%%%%%%%%%%%%%%%
\begin{comment}
\begin{columns}
\column{.65}
\block{More Examples of LateX}{
	\bigskip
	you can even put stuff in color boxes.\\
	\coloredbox{
		\begin{itemize}
			\item 1
			\item 2
		\end{itemize}
		\lipsum[1]
		\bigskip
		\innerblock{LINK:}{
			\center
			https://github.com/sergio1020881/PESTA2021
		}
	}	
}
\block{More text}{
	\lipsum[1]
}
\column{.35}
\block{more pictures}{
	\includegraphics[width=\linewidth]{./image/PESTA/ISEP_marca_cor_grande.png}
}
\block{the end}{
	\lipsum[1]
}
\block{referencces}{
%\printbibliography
}
\end{columns}
\end{comment}
%%%%%%%%%%%%%%%%%%%%%%%%%%%%%%%%%%%%%%%%%%%%%%%%%%%%%%%%%%%%%%%%