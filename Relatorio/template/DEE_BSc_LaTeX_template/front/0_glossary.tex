%----------------------------------------------------------------------------------------
%	GLOSSARY
%----------------------------------------------------------------------------------------

% Only the used entries will be displayed in the printed list, ie, you need to used a term at least once
% In italic if not in the main document language

% terms definition usage:
% \newglossaryentry{<tag>}{name={<term>},description={<description of the term>}}

\newglossaryentry{gloss}{
name={glossário}, 
description={é uma lista alfabética de termos de um determinado domínio de conhecimento com a definição desses mesmos termos.},
}

\newglossaryentry{pack}{
name={\textit{package}}, 
description={é um ficheiro ou conjunto de ficheiros que contêm comandos \LaTeX{} extra que adicionam novas funcionalidades de estilo ou modificam aquelas já existentes.},
sort={package}	%needed for sorting when using LaTeX commands in the 'name' field
}

\newglossaryentry{lipsum}{
name={\textit{Lorem Ipsum}}, 
description={é uma sequência de palavras, geralmente latinas, utilizada para preencher o espaço destinado a texto numa publicação, por forma a testar as opções de formatação e edição e o arranjo dos elementos gráficos antes da inserção do conteúdo.},
sort={Lorem Ipsum}
}
