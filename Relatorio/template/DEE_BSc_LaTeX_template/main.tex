%%%%%%%%%%%%%%%%%%%%%%%%%%%%%%%%%%%%%%%%%%%%%%%%%%%%%%%%%%%%%%%%%%%%%%%%%%%%%%%%%%%%%%%%%
% 
% Template for Project/Internship reports, for LEEC and LETI at DEE, ISEP,
% by Vitor M. R. Cunha - v1.0, May 2021
% Suggestions and comments are welcomed (vrc at isep dot ipp dot pt).
%
% Template provided as is, NO SUPPORT will be given.
% NO questions related to LaTeX will be answered, go to
% https://ftp.eq.uc.pt/software/TeX/info/lshort/english/lshort.pdf, or search the web.
%
% The DEE document class uses the LaTeX book base class, the original work credits are
% in the document class file.
%
% Overleaf template direct link:
% https://www.overleaf.com/latex/templates/isep-dee-bsc-latex-template/kqkmqnjbbpvj
% or go to https://www.overleaf.com/gallery/, and search for: ISEP LEEC
%
%%%%%%%%%%%%%%%%%%%%%%%%%%%%%%%%%% MAIN SETTINGS %%%%%%%%%%%%%%%%%%%%%%%%%%%%%%%%%%%%%%%%

\documentclass[
LEEC,			% Use this option to select your DEE degree, options: LEEC, LETI
portuguese,		% Select document language, options: portuguese, english
%draft,			% Uncomment for draft mode (no pictures, no links, overfull hboxes) 
]{DEEclass}

% Use the 'preamble.tex' file (root folder) do add packages and macros. Keep your main.tex file clean.

% FYI, the following packages are preloaded with the document class:
% longtable, xcolor, graphicx, booktabs, caption, csquotes, hyperref,
% calc, listings, datetime2, siunitx, geometry, enumitem

%%%%%%%%%%%%%%%%%%%%%%%%%%%%%%%%%%%%%%%%%%%%%%%%%%%%%%%%%% extra packages
\usepackage{amsmath}		% the principal package in the AMS-LATEX distribution
\usepackage{amsfonts}		% extended set of fonts for use in mathematics
\usepackage{amssymb}		% adds new symbols to be used in math mode
\usepackage{mathrsfs}		% math fonts, e.g., Laplace
\usepackage{float}			% provides the H float modifier option
\usepackage{multirow}		% tables \multirow command
\usepackage{subcaption}		% enables subfigures
\usepackage{lscape}			% for landscape mode
\usepackage{verbatim}		% new verbatim environment, \begin{comment}...\end{comment}, \verbatiminput

%add extra packages if needed here

%%%%%%%%%%%%%%%%%%%%%%%%%%%%%%%%%%%%%%%%%%%%%%%%%%%%%%%%%% temp packages
\usepackage{lipsum}						% for fake text
%\usepackage[textsize=tiny]{todonotes}   % enable To-do notes, use the option "disable" to hide all notes, usage \todo{}

%\usepackage{draftwatermark}			% prints a watermark overlay, uncomment if needed
%\SetWatermarkText{**DRAFT**}
%\SetWatermarkScale{1}
%\SetWatermarkColor[gray]{0.8}

%%%%%%%%%%%%%%%%%%%%%%%%%%%%%%%%%%%%%%%%%%%%%%%%%%%%%%%%%% settings
\AtBeginDocument{					% Rendered PDF metadata:
\hypersetup{pdftitle=\ttitle} 		% Sets the PDF title to your dissertation title
\hypersetup{pdfauthor=\authorname} 	% Sets the PDF author to your name
}

%%%%%%%%%%%%%%%%%%%%%%%%%%%%%%%%%%%%%%%%%%%%%%%%%%%%%%%%%% user defined macros

%....








		 

%%%%%%%%%%%%%%%%%%%%%%%%%%%%%%%% REPORT INFORMATION %%%%%%%%%%%%%%%%%%%%%%%%%%%%%%%%%%%%%

\reporttitle{Título do Relatório} % Your report title

%\reportsubtitle{Com um Subtítulo se Necessário} % and subtitle, uncomment if needed
%\subdate{Janeiro, 2021} % Uncomment for a static submission date, or leave it as a comment for automatic date (month+year) 

\author{Nome do Candidato}	% Your name
\studentnumber{1234567}	% Your student number
\studentemail{1234567@isep.ipp.pt}	% Your student email address  

\advisor{Nome do Orientador}{xxx@isep.ipp.pt}	% Your ISEP advisor name and email
\coadvisor{Nome do Coorientador}{xxx@isep.ipp.pt}	% Your ISEP co-advisor name and email, comment this line if not needed
\company{Nome da Empresa, Lda.}	% The company name where you developed your work, comment this line if not needed
\supervisor{Nome do Orientador da Empresa}{xxx@emailaddress.com} % Your company supervisor name, comment this line if not needed

%%%%%%%%%%%%%%%%%%%%%%%%%%%%% USER DEFINED LISTS %%%%%%%%%%%%%%%%%%%%%%%%%%%%%%%%%%%%%%%%
\makeglossaries						% Consider the following files in the 'front' folder:
%---------------------------------------------------------
%	GLOSSARY
%---------------------------------------------------------
% Only the used entries will be displayed in the printed list, ie, you need to used a term at least once
% In italic if not in the main document language
% terms definition usage:
% \newglossaryentry{<tag>}{name={<term>},description={<description of the term>}}
%\makeglossaries
\newglossaryentry{gloss}{
name={glossário}, 
description={é uma lista alfabética de termos de um determinado domínio de conhecimento com a definição desses mesmos termos.},
}
\newglossaryentry{pack}{
name={\textit{package}}, 
description={é um ficheiro ou conjunto de ficheiros que contêm comandos \LaTeX{} extra que adicionam novas funcionalidades de estilo ou modificam aquelas já existentes.},
sort={package}	%needed for sorting when using LaTeX commands in the 'name' field
}
\newglossaryentry{lipsum}{
name={\textit{Lorem Ipsum}}, 
description={é uma sequência de palavras, geralmente latinas, utilizada para preencher o espaço destinado a texto numa publicação, por forma a testar as opções de formatação e edição e o arranjo dos elementos gráficos antes da inserção do conteúdo.},
sort={Lorem Ipsum}
}
%%%%%%%%%%%%%%%%%%%%%%%%%%%%%%%%%%%%%%%%%%%%%%%%%%%%%%%%%%%%%%%%
\newglossarystyle{mylong}{%
	\setglossarystyle{long}%
	\renewenvironment{theglossary}%
	{\begin{longtable}[l]{@{}p{\dimexpr 3cm-\tabcolsep}p{.7\hsize}}}% <-- change the value here
		{\end{longtable}}%
}
\newglossaryentry{latex}
{
	name=latex,
	description={Is a mark up language specially suited 
		for scientific documents}
}
\newglossaryentry{maths}
{
	name=mathematics,
	description={Mathematics is what mathematicians do}
}
\newglossaryentry{formula}{
	name=formula,
	description={A mathematical expression}
}
\newglossaryentry{Biofouling}{
	name=Biofouling,description={Some description}
}
\newglossaryentry{symb:Pi}{
	name=\ensuremath{\pi},
	description={Geometrical value}
}
%%%%%%%%%%%%%%%%%%%%%%%%%%%%%%%%%%%%%%%%%%%%%%%%%%%%%%%%%%
			% Edit to define your glossary entries list
%----------------------------------------------------------------------------------------
%	ACRONYMS LIST
%----------------------------------------------------------------------------------------

% Only the used entries will be displayed in the printed list, ie, you need to used a acronym at least once
% Full name in italic if not in the main document language

%acronym definition usage:
%\newacronym{<tag>}{<acronym>}{<full name>}

\newacronym{wys}{WYSIWYG}{\textit{What You See Is What You Get}}
\newacronym{dee}{DEE}{Departamento de Engenharia Electrotécnica}
\newacronym{api}{API}{\textit{Application Programming Interface}}
\newacronym{ascii}{ASCII}{\textit{American Standard Code for Information Interchange}}
\newacronym{html}{HTML}{\textit{HyperText Markup Language}}
\newacronym{isep}{ISEP}{Instituto Superior de Engenharia do Porto}
\newacronym{leec}{LEEC}{Licenciatura em Engenharia Eletrot\'{e}cnica e de Computadores}
\newacronym{leti}{LETI}{Licenciatura em Engenharia de Telecomunicações e Informática}
\newacronym{usb}{USB}{\textit{Universal Serial Bus}}
\newacronym{pdf}{PDF}{\textit{Portable Document Format}}
			% Edit to define your acronyms entries list
%---------------------------------------------------------
%	SYMBOLS LIST
%---------------------------------------------------------
% terms definition usage:
% \newglossaryentry{<tag>}{
% name={<symbol>},
% sort={<text for the alphabetical sorting>},
% description={<description of the symbol>},
% unit={<units to display>},
% type=symbolslist}
\newglossaryentry{f}{
name={\ensuremath{f}},
sort={f},
description={força},
unit=\si{\newton},
type=symbolslist}
\newglossaryentry{i}{
name={\ensuremath{i}},
sort={i},
description={corrente},
unit=\si{\ampere},
type=symbolslist}
\newglossaryentry{m}{
name=\ensuremath{M},
sort={m},
description={massa},
unit=\si{\kilogram},
type=symbolslist}
\newglossaryentry{p}{
name=\ensuremath{P},
sort={p},
description={potência},
unit=\si{\watt},
type=symbolslist}
\newglossaryentry{theta}{
name=\ensuremath{\theta},
sort={z1},
description={deslocamento angular},
unit=\si{\radian},
type=symbolslist}
\newglossaryentry{omega}{
name=\ensuremath{\omega},
sort={z2},
description={velocidade angular},
unit=\si{\radian\per\second},
type=symbolslist}
\newglossaryentry{x}{
name=\ensuremath{x},
sort={x},
description={deslocamento},
unit=\si{\meter},
type=symbolslist}
% Greek: alpha,beta,gamma,delta,epsilon,zeta,eta,theta,iota,kappa,lambda,mu,nu,xi,omikron,pi,rho,sigma,tau,upsilon,phi,chi,psi,omega
%%%%%%%%%%%%%%%%%%%%%%%%%%%%%%%%%%%%%%%%%%%%%%%%%%%%%%%%%%
				% Edit to define your symbols entries list

%%%%%%%%%%%%%%%%%%%%%%%%%%%%%%%%%%%%%%%%%%%%%%%%%%%%%%%%%%%%%%%%%%%%%%%%%%%%%%%%%%%%%%%%%
\begin{document}
\frontmatter

%----------------------------------------------------------------------------------------
%	TITLE PAGES
%----------------------------------------------------------------------------------------
\pagestyle{plain} % Default to the plain heading style until the thesis style is called for the body content
\printcoverpage
\printaftercoverpage
\cleardoublepage















%%%%%%%%%%%%%%%%%%%%%%%%%%%%%%%%%%% FRONTMATTER %%%%%%%%%%%%%%%%%%%%%%%%%%%%%%%%%%%%%%%%%
% Consider the following front matter sections provided as separate files in the 'front' folder.
% Comment the lines regarding the sections you will not use, or edit the file contents as needed

%----------------------------------------------------------------------------------------
%	DEDICATION
%----------------------------------------------------------------------------------------

\dedicatory{
(Opcional) Poderá usar esta secção para dedicar o trabalho a alguém\ldots
} 
		% Edit if you want to dedicate your work to someone, or comment this line if not used
%---------------------------------------------------------
%	ACKNOWLEDGEMENTS
%---------------------------------------------------------
\begin{acknowledgements}
Ao \ac{isep}, como instituto com fortes valores e princípios, que permaneçam no bom caminho, criando cada vez mais formandos aptos para enfrentar os desafios do futuro, e crescer de forma positiva. Agradecer aos docentes que partilharam seus conhecimentos e nos permitiu desenvolver as competências necessárias para atingir os nossos objetivos. \\
\\
Um especial agradecimento ao orientador do Projeto/Estagio Engª Isabel Gonçalves Vaz, pelos conselhos e opiniões.
\end{acknowledgements}
%%%%%%%%%%%%%%%%%%%%%%%%%%%%%%%%%%%%%%%%%%%%%%%%%%%%%%%%%%
	% Edit to add the due acknowledgements, or comment this line if not used
%----------------------------------------------------------------------------------------
%	ABSTRACT PAGES
%----------------------------------------------------------------------------------------

% IMPORTANT NOTE: the abstract must always be written in two languages. If the report
% is written in Portuguese you have selected 'portuguese' as the language in the document class.
% Therefore, the portuguese version of the abstract must come first, so write it in the
% below area denoted by 'MAIN LANGUAGE ABSTRACT'. The english version follows in the
% 'SECOND LANGUAGE ABSTRACT' section.
% If the report is written in English, first will come the abstract in English
% ('MAIN LANGUAGE ABSTRACT') and then in Portuguese ('SECOND LANGUAGE ABSTRACT').

\begin{abstract}
%%%%%%%%%%%%%%%%%%%%%%%%%%%%%% MAIN LANGUAGE ABSTRACT %%%%%%%%%%%%%%%%%%%%%%%%%%%%%%%%%%

Aqui deverá ser apresentado o resumo de todo o trabalho efetuado. Esta secção não deverá exceder uma página.

Deve contextualizar o problema que pretende resolver ou a hipótese que irá formular, procure evidenciar as vantagens e desvantagens (se as houver) da solução encontrada, como também a forma através da qual a solução/hipótese foi validada. Neste último ponto, deverá referir-se aos desenvolvimentos efetuados, e à forma como validou (conformidade) e avaliou (desempenho) a solução encontrada.

O documento deve conter sempre duas versões do resumo: uma primeira no idioma do texto principal e a segunda num outro idioma. Este \textit{template} assume que os dois idiomas em consideração são sempre Português e Inglês, assim, a classe irá colocar os cabeçalhos respetivos de acordo com o idioma selecionado nas opções da classe no ficheiro \file{main.tex}. 

%----------------------------------------------------------------------------------------

\vspace*{10mm} 
\noindent
\textbf{\keywordslabel}: Lista, separada por vírgulas, de palavras, frases, ou acrónimos chave no âmbito do trabalho descrito neste texto. 

%%%%%%%%%%%%%%%%%%%%%%%%% END OF THE MAIN LANGUAGE ABSTRACT %%%%%%%%%%%%%%%%%%%%%%%%%%%%%%
\end{abstract}
\begin{secondlangabstract}
%%%%%%%%%%%%%%%%%%%%%%%%%%%%%% SECOND LANGUAGE ABSTRACT %%%%%%%%%%%%%%%%%%%%%%%%%%%%%%%%%%

The summary of all the developed work should be presented here. This section should not exceed one page.

Start the abstract with the contextualization of the problem you intend to solve or the hypothesis you will formulate. Try to highlight the advantages and disadvantages (if any) of the solution found, as well as the way in which the solution/hypothesis was validated. In this last point, you should refer to the developments made, and to the way you validated (compliance) and evaluated (performance) the solution found.

The document must always contain two versions of the abstract: a first in the language of the main text and the second one in another language. This template assumes that the two languages are always Portuguese and English, therefore, the class will place the correct section headers according to the language selected in the class options in the \file{main.tex} file.


%----------------------------------------------------------------------------------------

\vspace*{10mm} 
\noindent
\textbf{\keywordslabel}: Comma separated list of words, phrases, or key acronyms within the scope of your developed work. 

%%%%%%%%%%%%%%%%%%%%%%%%%% END OF THE SECOND LANGUAGE ABSTRACT %%%%%%%%%%%%%%%%%%%%%%%%%%%%%
\end{secondlangabstract}

			% Edit the file to write the document Abstract. Two languages are always required.
%----------------------------------------------------------------------------------------
%	FONTMATTER LISTS
%----------------------------------------------------------------------------------------
\pdfbookmark[0]{\contentsname}{toc}
\tableofcontents 	
\glsresetall
%----------------------------------------------------------------------------------------

% Of the following lists, comment the ones you will not use in your document

\listoffigures 			% Prints the list of figures

\listoftables 			% Prints the list of tables

\printlistoflistings	% Prints the list of listings (source code segments)

\printlistofterms		% Prints the list of USED terms (glossary)

\printlistofacronyms{XXXXXXXXI} % Prints the list of USED acronyms
% Change the argument with random letters to adjust the left alignment of the acronyms full name column

\printlistofsymbols		% Prints the list of ALL defined symbols
	% Edit the file to select the lists to be shown (figures, tables, source code segments, glossary, acronyms, symbols)

%%%%%%%%%%%%%%%%%%%%%%%%%%%%%%%%%%% MAINMATTER %%%%%%%%%%%%%%%%%%%%%%%%%%%%%%%%%%%%%%%%%
\mainmatter 
\pagestyle{thesis} 				

% Include the chapters of the document as separate files from the 'chapters' folder

%%%%%%%%%%%%%%%%%%%%%%%%%%%%%%%%%%%% Chapter Template

\chapter{Introdução} 	% Main chapter title
\label{Chapter1} 		% For referencing the chapter elsewhere, usage \ref{Chapter1}

%%%%%%%%%%%%%%%%%%%%%%%%%%%%%%%%%%%%

Este documento pretende guiar o Estudante na elaboração do relatório de Projeto/Estágio, do 3º ano da \ac{leec} e da \ac{leti}, do \ac{dee}, do \ac{isep}.

O autor deverá ter em consideração as seguintes regras gerais na elaboração do documento:

\begin{itemize}
	\item É fundamental refletir, antes de escrever, sobre a substância do que se pretende transmitir.
	\item Deve organizar o texto evitando a excessiva divisão do mesmo em tópicos que supostamente se enquadram no tema principal da secção a que pertencem. A excessiva especialização pode ser reveladora de falta de conhecimento e/ou reflexão. Neste sentido, deve, antes de iniciar a escrita, exercitar-se em refinar a organização do texto com o intuito de evitar que existam mais do que 2 níveis de ``profundidade'' em cada secção (respetivamente, subsecção e sub-subsecção). Note que o número de capítulos, secções e subsecções do presente documento não é vinculativo nem mesmo indicativo, apenas serve os propósitos do mesmo.
	\item O documento deve ser redigido em português ou inglês com um estilo adequado (evite o tom coloquial, lugares comuns e chavões) e correto do ponto de vista gramatical (quer do ponto de vista sintático quer semântico).
	\item Tenha especial cuidado com o uso de adjetivos (facilmente conduzem ao exagero), advérbios (nada, ou quase nada, acrescentam) e sinais de pontuação (em especial o uso correto das vírgulas).
	\item O estilo adotado para a redação deve ser coerente com as exigências de um trabalho científico encontrado em publicações impressas.
	\item De uma forma genérica deve usar a 3ª pessoa do singular (eventualmente do plural), exceção feita aos locais onde tal é claramente desajustado, por exemplo, na secção dos agradecimentos.
	\item Usar o estilo \textit{itálico} sempre que são utilizados termos em línguas diferentes da língua adotada no relatório.
	\item O uso de acrónimos implica que na 1ª vez que são utilizados se apresentem por extenso, colocando entre parênteses a respetiva sigla que se passará a usar. No entanto, é sempre possível, mais à frente no texto, e por uma questão de legibilidade, repetir o significado do acrónimo. Todos os acrónimos devem ser apresentados por ordem alfabética na secção ``Lista de Acrónimos''.
	\item O uso correto de unidades, seus múltiplos e submúltiplos.
	\item As imagens e tabelas devem, por princípio, aparecer no topo ou no fundo da página. A legendas surgem imediatamente após as figuras e listagens. No caso das tabelas as legendas antecedem as mesmas.
	\item Todas as figuras, tabelas e restantes listagens devem ser mencionadas no texto por forma a que fiquem enquadradas nas ideias transmitidas pelo autor. Esta referência, regra geral, deverá ser feita antes da ocorrência da figura, tabela ou listagem.
	\item Deve indicar ao longo do texto as referências documentais usadas, em especial nas citações (puras ou literais), assinaladas com a utilização de aspas, como também no caso de reutilização de gráficos, figuras, tabelas, fórmulas, etc., de outras fontes.
\end{itemize}

De uma forma já mais específica, neste primeiro capítulo obrigatório (``Introdução'') o autor deve:

\begin{itemize}
	\item contextualizar a proposta de trabalho no âmbito da empresa, de um outro trabalho já realizado, do ponto de vista científico e/ou tecnológico, etc.,
	\item apresentar de forma clara os objetivos que se propõe atingir,
	\item descrever de forma sucinta, mas objetiva, a solução preconizada ou a hipótese colocada,
	\item apresentar de forma resumida, mas clara, os desenvolvimentos efetuados,
	\item identificar como foi validada e avaliada a solução encontrada,
	\item descrever a organização do documento.
\end{itemize}

Sem ser exigida nenhuma organização em particular para este capítulo, são indicadas a título de exemplo 4 secções (com texto \gls{lipsum}) que podem ser incorporadas nesta parte do documento: Contextualização, Descrição do Projeto, Calendarização e Organização do Relatório. 	

%%%%%%%%%%%%%%%%%%%%%%%%%%%%%%%%%%%% SECTION 1

\section{Contextualização}
\label{sec:Ch1.1}

\lipsum[1] % inserts fake text, to be removed in your dissertation

%%%%%%%%%%%%%%%%%%%%%%%%%%%%%%%%%%%% SECTION 2

\section{Descrição do Projeto}
\label{sec:Ch1.2}

\lipsum[2]

%%%%%%%%%%%%%%%%%%%%%%%%%%%%%%%%%%%% SUBSECTION 1

\subsection{Objetivos}
\label{sub:Ch1.2.1}

\lipsum[2]

%%%%%%%%%%%%%%%%%%%%%%%%%%%%%%%%%%%% SECTION 3

\section{Calendarização}
\label{sec:Ch1.3}

\lipsum[1]
% you could use the pgfgantt package: https://ctan.org/pkg/pgfgantt


%%%%%%%%%%%%%%%%%%%%%%%%%%%%%%%%%%%% SECTION 4

\section{Organização do Relatório}
\label{sec:Ch1.4}

\lipsum[2]




\chapter{Balança Digital}
\label{Chapter2}
Para o desenvolvimento deste projeto, foi criado um kit de desenvolvimento para facilitar sua implementação, testar, efetuar alterações e melhoramentos.
\\
\\
Abaixo pode-se ver a montagem em esqueleto do equipamento utilizado na \textit{figura} \ref{Kit_Desenvolvimento_2},
\begin{figure}[H]
	\centering
	\includegraphics[scale=0.12]{./image/PESTA/kit/Kit_Desenvolvimento_2.jpg}
	\caption{Kit de Desenvolvimento}
	\label{Kit_Desenvolvimento_2}
\end{figure}
a seguir a \textit{figura} \ref{Block_diagram_1} representado os elementos em diagrama de blocos.
\begin{figure}[H]
	\centering
	\includegraphics[scale=0.27]{./image/PESTA/Diagrama/Diagrama_bloco_3.jpg}
	\caption{Diagrama Blocos}
	\label{Block_diagram_1}
\end{figure}
\section{sensor}
Para medir a massa recorreu-se a uma \textbf{célula de peso} que determina a pressão exercida por um dado objeto, neste caso é um bloco de alumínio como indicado na \textit{figura} \ref{Load_Cell_1}, para isso ser possível este utiliza sensores Piezoresistivos numa montagem em ponte \textit{wheatstone} sobre essa superfície em locais determinados.
\\
\begin{figure}[H]
	\captionsetup{justification=raggedright,singlelinecheck=false}
	\flushleft
	\includegraphics[scale=0.15]{./image/PESTA/material/Load_Cell_1.jpg}
	\caption{Célula de Carga 50Kg}
	\label{Load_Cell_1}
\end{figure}
\figurespace{.5}
Piezoresistividade deriva seu nome da palavra grega \textit{piezin}, que significa "pressionar". É um efeito exibido por vários materiais que exibem uma mudança na resistividade devido a uma pressão aplicada. O efeito foi descoberto pela primeira vez por Lord Kelvin em \textcolor{blue}{1856}, que notou que a resistência dos fios de cobre e ferro aumentava quando em tensão. Ele também observou que os fios de ferro apresentavam uma alteração maior na resistência do que os de cobre. A primeira aplicação do efeito piezoresistivo não apareceu até a década de \textcolor{blue}{1930}, cerca de \textcolor{blue}{75} anos após a descoberta de Lord Kelvin. Em vez de usar fios de metal, esses assim chamados medidores de tensão são geralmente feitos de uma folha de metal fina montada em uma película de suporte, que pode ser colada em uma superfície. O sensor de fita de metal típico é representado na \textit{figura} \ref{strain_gauge_1} \cite{book-9}.
\\
\\
\begin{minipage}[!b]{.5\linewidth}
	\begin{figure}[H]
		\captionsetup{justification=raggedright,singlelinecheck=false}
		\flushleft
		\includegraphics[height=5cm]{./image/PESTA/general/strain_gauge_1.jpg}
		\caption{Fita metálica \textit{strain gauge} \cite{book-9}}
		\label{strain_gauge_1}
	\end{figure}
\end{minipage}
\begin{minipage}[!b]{.5\linewidth}
	\begin{figure}[H]
		\captionsetup{justification=raggedright,singlelinecheck=false}
		\flushleft
		\includegraphics[height=5cm]{./image/PESTA/schematic/Wheatstone_2.jpg}
		\qquad \caption{Ponte \textit{Wheatstone}}
		\label{wheatstone_2}
	\end{figure}
\end{minipage}
\minipagespace{.5}
\begin{minipage}[!b]{.4\linewidth}
	\begin{figure}[H]
		\captionsetup{justification=raggedright,singlelinecheck=false}
		\flushleft
		\includegraphics[height=5cm]{./image/PESTA/schematic/Wheatstone_1.jpg}
		\caption{\textit{Wheatstone} por resistências \cite{book-10}}
		\label{wheatstone_1}
	\end{figure}
\end{minipage}
\begin{minipage}[!b]{.6\linewidth}
	\begin{align}
		\label{eq:wheatstone}
		&V_A =  \frac{R_2}{R_1 + R_2} \; V_{ref} \; V_B=\frac{R_4}{R_3 + R_4} \; V_{ref} \\
		&V_{AB} =  V_A - V_B = e \\
		&V_{AB}= \left(\frac{R_2}{R_1 + R_2} - \frac{R_4}{R_3 + R_4}\right) \; Vref \\
		&e = \frac{R_2 R_3 - R_4 R_1}{(R_1 + R_2)(R_3 + R_4)} \; Vref
	\end{align}
	\minipagespace{.1}
\end{minipage}
\minipagespace{.5}
Normalmente nestas aplicações só é usados um sensor ou dois sensores em que estão nos extremos opostos  ou ligados ao mesmo ponto da alimentação, só em casos muito raros são utilizados os quatro sensores na qual a sensibilidade é máxima. E como é óbvio se o valor das quatro resistências são iguais $V_{AB}$ na saída é nula, e quando se utiliza apenas dois sensores a sensibilidade do sistema é intermédia.
\newpage
A montagem da mesa de medição \textit{figura} \ref{Prato},
\minipagespace{5}
\begin{minipage}[!b]{\linewidth}
	\begin{figure}[H]
		\centering
		\includegraphics[scale=0.16]{./image/PESTA/material/Prato.jpg}
		\caption{Prato}
		\label{Prato}
	\end{figure}
\end{minipage}
\newpage
\section{Amplificador de sinal}
A amplificação é geralmente um requisito fundamental, pois a maioria dos sensores tende a produzir níveis de sinal significativamente mais baixos do que aqueles usados no processador digital. Sensores resistivos podem precisar de um amplificador de carga. Se possível, é vantajoso ter o ganho o mais próximo possível do elemento sensor. Em situações onde um alto ganho é necessário, muitas vezes pode haver implicações para lidar 
com quaisquer efeitos adversos, como o ruído, também em termos de \textit{layout} do \textit{chip}, os transitórios agudos associados aos sinais digitais precisam ser mantidos bem longe dos circuitos analógicos \textit{front-end}. \cite{book-9}
\\
\\
A ligação destes componentes é intuitivo e fácil de se perceber, o que é complexo neste trabalho é a interligação destes equipamentos com o micro-controlador por meio de \textit{software} e criar o \textit{driver} de comunicação para a placa do amplificador de sinal, já que o protocolo de comunicação é proprietário.
\\
\\
\begin{figure}[H]
	\captionsetup{justification=raggedright,singlelinecheck=false}
	\centering
	\includegraphics[scale=0.35]{./image/PESTA/schematic/HX711_Schematic_1.jpg}
	\includegraphics[scale=0.1]{./image/PESTA/material/HX711_board_1.jpg}
	\caption{Amplificador de Sinal [HX711]}
	\label{HX711_Schematic_1}
\end{figure}
\figurespace{.5}
A placa \textit{Load Cell Amplifier }pode ser programada fisicamente para determinar o numero de amostras por segundo a ser transmitido, tem opção de \textcolor{blue}{10} amostras por segundo e \textcolor{blue}{80} amostras, neste projeto optei pela segunda opção que necessita alteração na placa de circuito de impresso, isto é, abrir o \textit{jumper} respetivo de configuração.
\\
\\
\begin{table}[H]
	\centering
	\caption{Terminais HX711 ({\tiny \scriptsize{top view}})}
	\begin{tabular}{||L{1cm} C{3cm} | p{3cm}  C{2cm}||}
		\hline
		\multicolumn{2}{||c|}{MCU} & \multicolumn{2}{|c||}{\textit{Célula de peso}}\\ [1ex]
		\hline
		1 & GND & EARTH (GND) & YLW \\ 
		2 & CLK & INPA & GRN \\
		3 & DATA & INNA & WHT \\
		4 & VCC &  GND & BLK \\
		5 & VDD & $V_{ADC}$ & RED \\ [1ex]
		\hline
	\end{tabular}	
	\label{HX711_connection}
\end{table}
\tablespace{.5}
A conversão de informação é a transição entre o sinal continuo da vida real para um sinal discreto associado ao mundo digital, tipicamente esta etapa consiste na conversão analógica para digital.
\\
O processamento digital pode consistir de rotinas para compensar os desvios por linearização, compensação da sensibilidade e \textit{offset}, ou podem ser técnicas mais sofisticadas como reconhecimento de padrões (tais como redes neuronais) para equipamentos de sensores vetoriais.\cite{book-9}
\\
A comunicação trata de cuidar das rotinas necessárias para transferir e receber a informação e sinais de controle para a linha de comunicação com o sensor, e o processador que toma lugar como componente central tratando a informação, guardar os dados e fazer rotinas tais como de calibração, teste e controlo de ganho da amplificação. \cite{book-9}
\\
\\
\begin{figure}[H]
	\centering
	\includegraphics[scale=0.55]{./image/PESTA/graph/80SPS64GAIN/SPS_80.JPG}
	\caption{Amostras}
	\label{SPS_64}
\end{figure}
\figurespace{.5}
A livraria (\textit{driver}) criada recorre a interrupções periódicas quando o sinal de \textit{data} vai para a massa, indicando assim que tem um pacote de leitura pronto a ser transmitido.
\\
\\
\begin{minipage}[!b]{.40\linewidth}
	\begin{table}[H]
		\captionsetup{justification=raggedright,singlelinecheck=false}
		\caption{Configuração Ganho}
		\begin{tabular}{ | c | c | c |  }
			\hline
			\makecell[c]{PD\_SCK \\ Impulsos} & Entrada  & Ganho \\
			\hline
			\hline
			25 & \textbf{A} & 128 \\
			\hline
			26 & \textbf{B} & 32 \\
			\hline
			27 & \textbf{A} & 64 \\
			\hline
		\end{tabular}
		\label{Gain_Selection}
	\end{table}
	\tablespace{2}
\end{minipage}
\begin{minipage}[l]{.6\linewidth}
	\vspace{.3cm}
	Como indicado abaixo no gráfico em que a linha \textcolor{yellow}{amarela} é a informação e a linha \textcolor{BlueGreen}{azul} o respetivo \textit{clock} que é gerado pelas interrupções do micro-controlador fazendo \textit{shift} dos \textcolor{blue}{24} bits, que depois no fim transmite para o amplificador o ganho de amplificação a ser usado pelo numero excedente de \textit{clock cycles}, em que nesta demonstração \textit{figura} \ref{Gain_128_example} é \textcolor{blue}{um}, e corresponde a ganho de \textcolor{blue}{128}, respeitando a \textit{tabela} \ref{Gain_Selection},  e a sequir o exemplo da \textit{figura} \ref{Gain_64_example} com o ganho de \textcolor{blue}{64}, pois tem \textcolor{blue}{três} impulsos excedentes.
	\\
\end{minipage}
\minipagespace{.5}
\begin{minipage}[!b]{.5\linewidth}
	\begin{figure}[H]
		\captionsetup{justification=raggedright,singlelinecheck=false}
		\flushleft
		\includegraphics[scale=0.25]{./image/PESTA/graph/80SPS128GAIN/Gain_128_example.JPG}
		\caption{Ganho de 128}
		\label{Gain_128_example}
	\end{figure}
\end{minipage}
\hspace{1cm}
\begin{minipage}[!b]{.5\linewidth}
	\begin{figure}[H]
		\captionsetup{justification=raggedright,singlelinecheck=false}
		\flushleft
		\includegraphics[scale=0.25]{./image/PESTA/graph/80SPS64GAIN/Gain_64_example.JPG}
		\caption{Ganho de 64}
		\label{Gain_64_example}
	\end{figure}
\end{minipage}
\minipagespace{.5}
Para obter este resultado a livraria driver para o \textit{Load Cell Amplifier} teve de ter em consideração que o micro-controlador é de \textcolor{blue}{8} bits, porque o pacote de informação consiste de \textcolor{blue}{24} \textit{bits} e em que é transmitido primeiro o \ac{msb}.
\\
\\
\\
O código que executa esta rotina é demonstrado na \textit{lista} \ref{HX711-read-raw} que é chamada pelas interrupções periódicas e só é ativa quando a função na \textit{lista} \ref{Main-While-case-1} \textbf{hx.query(\&hx)} é verdadeira.
\\
{ \tiny
	\lstinputlisting[language=C, caption={função de chamada \textbf{hx.query(\&hx)}}, captionpos=b, label=Main-While-case-1]{./input/code/hx_query.c}
}
\newpage
{ \tiny
	\lstinputlisting[language=C, caption=HX711\_read\_raw, captionpos=b, label=HX711-read-raw]{./input/code/HX711_read_raw.c}
}
\listingspace{.1}
Após obter um numero determinado de valores discretos é calculado sua média
\begin{equation}
	\label{eq:Mean}
	\overline{x}  =  \frac{1}{n}\sum_{i=1}^n x_i
\end{equation}
para ser tratado e deduzido o valor correspondente da massa.
\newpage
\section{Display LCD}
O \ac{lcd} utilizado é de 4x20, isto é, quatro linhas de vinte caracteres cada, é o interface humano principal, e durante o projeto uma ferramenta extremamente útil também para fazer \textit{debug} e executar testes no código.
\\
\\
Uma livraria na qual já tinha feito para outros projetos serviu para aplicar neste, poupando bastante tempo, revelando a importância de documentar os conhecimentos adquiridos. A livraria ou se preferem \textit{driver} esta \textit{anexado}.
%[\ref{codigo}]
\\
\\
Abaixo esta uma tabela \ref{LCD_connections} com as respetivas ligações.
\tablespace{.2}
\begin{table}[H]
	\centering
	\caption{Conexões \textbf{LCD}}
	\begin{tabular}{||p{1cm} p{2cm} p{4cm} | p{1cm}||} 
		\hline
		\multicolumn{3}{||c|}{\textbf{LCD Pin}} & \multicolumn{1}{|c||}{\textbf{MCU Pin}}\\ [1ex]
		\hline
		1 & VSS & GND & \\
		2 & VCC & +5V & \\
		3 & VEE & \textit{Contrast Control} & \\
		4 & RS & \textit{Register Select} & Pin 0 \\
		5 & RW & \textit{Read/Write} & Pin 1 \\
		6 & E & \textit{Enable} & Pin 2 \\
		7 & Do & \textit{Data Pin 0} & \\
		8 & D1 & \textit{Data Pin 1} & \\
		9 & D2 & \textit{Data Pin 2} & \\
		10 & D3 & \textit{Data Pin 3} & \\
		11 & D4 & \textit{Data Pin 4} & Pin 4 \\
		12 & D5 & \textit{Data Pin 5} & Pin 5 \\
		13 & D6 & \textit{Data Pin 6} & Pin 6 \\
		14 & D7 & \textit{Data Pin 7} & Pin 7 \\
		15 & LED+ & \textit{Led +5V} &  \\
		16 & LED- & \textit{Led Ground} & \\
		\multicolumn{3}{||c|}{\textit{Reboot} LCD} & \multicolumn{1}{|l||}{Pin 3}\\ [1ex]
		\hline
	\end{tabular}	
	\label{LCD_connections}
\end{table}
\begin{figure}[H]
	\centering
	%%\captionsetup{justification=raggedright,singlelinecheck=false}
	\includegraphics[scale=.5]{./image/PESTA/material/4x20_LCD.jpg}
	\caption{LCD}
	\label{4x20_LCD}
\end{figure}
\newpage
\section{Micro-controlador}
Os micro-controladores da \textbf{Atmel} de 8 e 32 bits são baseados na arquitetura avançada de \textbf{Harvard} na qual esta concebido para baixos consumos e performance.
\\
\\
Este tipo de arquitetura tem dois  \textit{busses} (barramentos) um dedicado a leitura das instruções a executar e outra para escrita e leitura de \textit{data} (informação ou dados), isto assegura que uma nova instrução pode ser executada em cada ciclo de relógio, na qual elimina estados de espera quando não ha instruções prontas a executar.
\\
\\
Nos microcontroladores da \textbf{AVR} os barramentos estão configurados de forma a dar prioridade ao barramento das instruções do \ac{cpu} acesso a memoria flash enquanto o barramento da CPU de dados tem prioridade de acesso a \ac{sram}.
\\
\\
O espaço de memoria de dados é dividida em três, os \ac{gpr} as \ac{sfr} ou memoria de I/O e a \textit{data} \textbf{SRAM}.
\\
\\
Os microcontroladores da \textbf{AVR} utiliza uma arquitetura de instruçõeses \ac{risc} na qual reduz a complexidade dos circuitos na codificação de cada instrução.
\\
\\
Dai que os micro-controladores que se baseiam nestes tipos de arquitetura são sinonimo de código reduzido, alta performance e baixo consumo energético.
\\
\\
\begin{figure}[H]
	\centering
	%%\captionsetup{justification=raggedright,singlelinecheck=false}
	\includegraphics[scale=1]{./image/PESTA/Diagrama/Harvard_architecture.jpg}
	\caption{Harvard Architecture}
	\label{Harvard_architecture}
\end{figure}
\qquad link: \url{https://en.wikipedia.org/wiki/Harvard_architecture}
\newpage
Neste projeto apostei no Atmega 128 (\textit{figura} \ref{Atmega_128_pinagem}) por ser um dos mais poderosos \ac{mcu} da linha de 8 b\textit{bit} da Atmel, também por estar integrado já numa placa de desenvolvimento usando o sistema por fichas \ac{idc} que prefiro, ou seja, usando \textit{flatcables} para ligar os periféricos, e considero muito mais pratico do que sistema que esta na moda, tais como a linha Arduino, STMicroelectronics e a \ac{pic} por ter um interface por pinos.
\\
\\
\\
\\
\\
\begin{figure}[H]
	\centering
	%%\captionsetup{justification=raggedright,singlelinecheck=false}
	\includegraphics[scale=0.7]{./image/PESTA/material/Atmega128_1.jpg}
	\caption{Atmega 128}
	\label{Atmega_128_pinagem}
\end{figure}
\figurespace{1}
{Transparências Sistemas Digitais 2 \quad \textbf{ISEP} \quad 2008/2009 \quad \textit{link}}:
\\
\url{https://drive.google.com/file/d/1wgOGf8WwYY0OzDhRca9ypXz-iBO55YOF/view?usp=sharing}
\newpage
A características do micro-controlador ATmega 128 esta abaixo indicados na lista, os temporizadores considero o mais importante, o \ac{adc} nem vai ser usado neste projeto a alternativa é muito melhor, talvez os MCU nem deviam ter esta funcionalidade e realçar em meios de comunicação e memoria com as livrarias já disponíveis, este integrado é perfeito para esta aplicação em causa, sempre que fazemos projetos também temos de considerar os MCU de 32 \textit{bit} mas existe uma linha muito fina de apostar noutra alternativa e trabalhar com um sistema operativo devido a complexidade exigida, muito mais configurações e opções disponíveis do que um micro-controlador de 8 e 16 \textit{bit}.
\\
\\
\\
\begin{minipage}{\linewidth}
	{\Large Caracteristicas do Atmega 128 :}
	\begin{itemize}	
		\setlength\itemsep{-0.3em}
		\item Arquitectura RISC
		\item 33 instruções (a maior parte executada num único ciclo de execução)
		\item 32 x 8 registos de trabalho (arquitectura de registos)
		\item Até 16 MIPS (@16MHz) – 62.5ns / instrução
		\item 64K x 16 palavras de programa – 128K bytes FLASH
		\item 4K bytes de RAM interna
		\item 4K bytes de E2PROM de dados
		\item Ciclos de escrita / leitura – FLASH=10000, E2PROM=100000
		\item 7 Portos de IO \\
		\hspace*{.5cm}	-> 6 x 8 bits (Portos A .. F) \\
		\hspace*{.5cm}	-> 1 x 5 bits (Porto G)
		\item 2 x Timer / Counter de 8 bits
		\item 2 x Timer / Counter de 16 bits
		\item 1 x Real Time Counter ( com oscilador independente)
		\item 2 x PWM de 8 bits
		\item 6 x PWM de 16 bits
		\item ADC de 10 bits (8 canais)
		\item 2 x USART
		\item SPI
		\item TWI (I2C)
	\end{itemize}
\end{minipage}
\minipagespace{0.5}
\newpage
Para programar este microcontrolador (\textbf{Atmega 128}) foi utilizado o programador da marca da Atmel precisamente o \ac{atmel-ice} \textit{figura} \ref{Programador_1}, que para este equipamento tem disponível programação via \ac{isp} \textit{figura} \ref{ISP_6_8_10pin} e \ac{jtag}.
\minipagespace{.2}
\begin{minipage}[!b]{.5\linewidth}
	\begin{figure}[H]
		\captionsetup{justification=raggedright,singlelinecheck=false}
		\flushleft
		\includegraphics[scale=0.75]{./image/PESTA/programador/Atmel_ice.png}
		\caption{KIT ATMEL-ICE}
		\label{Programador_1}
	\end{figure}
\end{minipage}
\hspace{.5cm}
\begin{minipage}[!b]{.5\linewidth}
	\begin{figure}[H]
		\captionsetup{justification=raggedright,singlelinecheck=false}
		\flushleft
		\includegraphics[scale=0.45]{./image/PESTA/programador/isp_6pin.png}
		\hspace{.3cm}
		\includegraphics[scale=0.5]{./image/PESTA/programador/isp_8e10pin.png}
		\caption{Fichas \textbf{ISP}}
		\label{ISP_6_8_10pin}
	\end{figure}
\end{minipage}
\section{Fonte de Alimentação}
\begin{minipage}[!b]{.5\linewidth}
	\begin{figure}[H]
		\captionsetup{justification=raggedright,singlelinecheck=false}
		\flushleft
		\includegraphics[scale=0.5]{./image/PESTA/material/DCDC_converter.jpg}
		\caption{\textit{Buck Converter}}
		\label{buck-converter}
	\end{figure}
	\figurespace{.1}
\end{minipage}
\begin{minipage}[!b]{.5\linewidth}
	\small
	Caracteristicas DC/DC:
	\begin{itemize}
		\setlength\itemsep{-0.5em}
		\footnotesize
		\item 5A 75W Conversor Abaixador (\textit{Step-down})
		\item Alimentação Entrada: 4 - 38VDC
		\item Tensão de saída: 1.25 - 36VDC
		\item Corrente saída: 0 - 5A
		\item Potência saída: 75W
		\item Voltímetro: 4 até 40V, erro ±0.1V
		\item \textit{Led} indicadores
		\item Frequência de operação: 180KHz
		\item Eficiência até 96 \%
		\item Proteção Térmica
		\item Limitador de Corrente
		\item Proteção contra curto circuito
		\item NOTA: Não tem proteção de inversão de polaridade
		\item L x W x H = 6.6 x 3.9 x 1.8 CM
		\item Massa: 28g
	\end{itemize}
\end{minipage}
Como a \textit{motherboard} tem um regulador de tensão de 5 Volt para aumentar a eficiência  e ter uma alimentação variável de entrada utilizo o \textit{Buck Converter} da \textit{figura} \ref{buck-converter}.
\chapter{Software}
\label{Chapter3}
O \ac{ide} utilizado neste trabalho foi o \textbf{\textit{{Microchip Studio for AVR\textsuperscript{\textregistered} and SAM Devices}}} (\textit{version: 7.0.2542}). A programação foi feita em Linguagem \textbf{C}, sua estrutura sintática esta abaixo mencionado:
\\
\\
\begin{figure}[H]
	\centering
	\includegraphics[scale=0.6]{./image/PESTA/flowchart/Main_Program_1.jpg}
	\caption{Estrutura do Programa (fluxograma)}
	\label{Main_Program_1}
\end{figure}
\figurespace{.5}
O \textit{PROGRAM 1} é onde corre o programa da balança, e o \textit{PROGRAM 2} usado para calibração do \textit{Gain Factor}.
\\
\\
Todos os programas sequem uma estrutura sintática recursiva usando o seguinte modelo.
\\
\\
\begin{figure}[H]
	\centering
	\includegraphics[scale=0.40]{./image/PESTA/flowchart/Generic_structure.jpg}
	\caption{Sintaxe Genérica dos programas (fluxograma)}
	\label{Geneic_structure}
\end{figure}
\figurespace{.5}
Duas interrupções periódicas estão sempre a correr em \textit{background}, uma para fazer o \textit{shift} dos \textit{bit´s} da conversão \textbf{ADC} feita pelo amplificador de sinal HX711 e outra interrupção periódica de segundo em segundo usado para saltar de \textit{Menu} pelos botões.
\\
\\
\begin{minipage}{\linewidth}
	\begin{minipage}{.5\linewidth}
		\begin{figure}[H]
			\centering
			\includegraphics[scale=0.7]{./image/PESTA/flowchart/Interrupt_1.jpg}
			\caption{\textbf{ADC} conversão}
			\label{Interrupt_1}
		\end{figure}
	\end{minipage}
	\begin{minipage}{.5\linewidth}
		\begin{figure}[H]
			\centering
			\includegraphics[scale=0.7]{./image/PESTA/flowchart/Interrupt_2.jpg}
			\caption{Saltar de \textit{Menu}}
			\label{Interrupt_2}
		\end{figure}
	\end{minipage}
\end{minipage}
\minipagespace{.5}
Consultar código para leitura das rotinas de interrupção nas folhas \textit{anexas}.
\\
\\
\begin{minipage}{.40\linewidth}
	\begin{figure}[H]
		\flushleft
		\captionsetup{justification=raggedright,singlelinecheck=false}
		\includegraphics[scale=0.9]{./image/PESTA/Code/Livrarias.jpg}
		\caption{Livrarias}
		\label{Livrarias}
	\end{figure}
\end{minipage}
\begin{minipage}{.6\linewidth}
	Ao lado esta as livrarias usadas neste projeto, as que estão dentro da caixa amarela são as que foram criadas.
	A filosofia usada é de criar objetos que representam o hardware para o poder manipular via código. Como se pode observar foi criado uma livraria para os temporizadores, outra para as interrupções e \ac{eeprom}, depois criado livrarias para os componentes externos, isto é, o \textbf{LCD} e o integrado \textbf{HX711}. \\
	Uma abstração que torna simples executar qualquer algoritmo ou projeto, e isto só é possível depois de ultrapassar a barreira árdua e dolorosa de desenvolver as livrarias.
	\\
	\\
	\\
\end{minipage}
\minipagespace{.5}
Durante este projeto tentou-se dentro do possível sempre seguir as boas praticas de programação, como a \texttt{indentação} especifica para cada situação, e manter a estrutura de norma. Deve-se manter uma metodologia de trabalho que segue as normas assim é percetível para todos e facilita detetar \textit{bugs}, e é uma pratica que abrange todas as linguagens, por exemplo o \textbf{Python} é fundado nesse principio.
\\
\\
Quanto as interrupções é sempre um desafio, porque as tarefas tem que ser bem organizadas temporalmente para não entrar em conflitos, pretende-se assim sempre que o código corra na função \textit{main} e as interrupções algo esporádico e muito rápido, estas devem servir de \textit{flags} para executar rotinas na função \textit{main}, ou seja, servirem de \textbf{\textcolor{green}{INPUTS}}.
\newpage
\section{Validação}
%%%To validate is to justify why the choices made and alternatives that could be chosen.
As escolhas feitas estão dentro dos parâmetros da oferta disponibilizada. Apenas o conhecimento adquirido ao aprofundar o funcionamento dos componentes é o ganho mais evidente, facilitando a interpretação de situações e deteção de anomalias (\textit{troubleshooting}), derivado aos custos dos materiais serem caros.\\
\\
Apostei na marca \textbf{Atmel} devido a experiência e conhecimentos já adquiridos, se aposta-se noutra marca teria de enfrentar uma curva de aprendizagem e adaptação que no final a nível de custos beneficio seria desfavorável, pelo tempo a dispensar e de ser muito trabalhoso a refazer tudo novamente noutra arquitetura.\\
\\
O sensor usado é o mais comum nesta pratica, e escolha demonstrada, o circuito de interface é indiferente a escolha apenas é baseada na sua precisão, ou seja, é de \textcolor{blue}{24} \textit{bit} enquanto o \textbf{ADC} do \textbf{MCU} de \textcolor{blue}{10} \textit{bit}.
\\
\\
\\
\subsection{Material}
Abaixo esta indicado uma tabela dos materiais usados, assim como os preços.
\\
\\
\begin{table}[H]{
		\caption{Lista de material}
		\rowcolors{3}{blue!80!yellow!50}{blue!70!yellow!40}
		\begin{tabular}{ |p{12cm}|c|p{2cm}|  }
			\hline
			\multicolumn{3}{|c|}{Lista de Material} \\
			\hline
			Peça & Quant & Preço [uni] \\
			\hline
			Fonte de alimetação 12V 1A & 1 & \EUR{3.87} \\
			Conversor DC-DC com voltímetro & 1 & \EUR{7.75} \\
			ET BASE AVR Atmega128 Board & 1 & \EUR{23.92} \\
			Test Input Board  & 1 & \EUR{3.71} \\
			Test Output Board & 1 & \EUR{3.71} \\
			IDC Socket 10 way    & 12 & \EUR{0.31} \\
			IDC Header Straight 10 way    & 12 & \EUR{0.25} \\
			Flatcable    & ? & \EUR{?} \\
			20x4 LCD Module Blue & 1 & \EUR{12.24} \\
			SparkFun Load Cell Amplifier HX711 & 1 & \EUR{13.04}   \\
			50Kg Load Cell & 1 & \EUR{12} \\
			\hline
			& \textit{total} & \EUR{86.96} \\
			\hline
		\end{tabular}
	}
	\label{material}
\end{table}
\newpage
\subsection{Testar}
Quanto a funcionalidade no seu todo a balança tem \textcolor{blue}{quatro} botões e \textcolor{blue}{três} \textit{leds} ativados, um botão para fazer o \textit{offset} no \textcolor{green}{PORTF 0}, e dois botões com dupla função, fazer \textit{reset} para \textit{default} e incrementar, outro para entrar no menu de calibração e decrementar, o \textcolor{blue}{quarto} botão é reservado para \textit{enter} e assumir o valor introduzido na calibração.\\
\\
O botão \textcolor{green}{PORTF 3} quando premido durante \textcolor{blue}{cinco} segundos faz um \textit{reset} para configuração \textit{default} depois de o \textit{led} no \textcolor{red}{PORTC 6} piscar \textcolor{blue}{quatro} vezes.\\
\\
O botão \textcolor{green}{PORTF 4} quando premido durante \textcolor{blue}{cinco} segundos entra no menu de calibração do valor do \textit{gain factor} e o \textit{led} no \textcolor{red}{PORTC 7} liga, usando os botões de incrementa e decrementar, isto é, o
\textcolor{green}{PORTF 3} e \textcolor{green}{PORTF 4} pode-se alterar esse valor.\\
\\
Para assumir o valor e sair do menu de calibração basta premir o botão colocado no \textcolor{green}{PORTF 5}. Tanto no caso de calibração ou de \textit{offset} os valores são guardados na \textbf{EEPROM} do microcontrolador, sendo que, se retirar a alimentação do circuito este não perde os valores e o \textit{led} \textcolor{red}{PORTC 5} permanece ligado.
\\
\\
%%%%%%%%%%%%%%%%%%%%%%%%%%%%%%%%%%%%%%%%%%%%%%%%%%%%%%%%%%%%%%%%
\begin{comment}
Sem contar com as despesas no equipamento para a programação do hardware que em principio só se gasta uma vez, isto é, se não se estragar. No caso do programador \textbf{Atmel-ICE} pode custar até \EUR{185.55}.\\
\\
É de ter em conta que os preços são \textbf{PVP}, que no caso se for preços comerciais são dez vezes inferior, e se for para produção em grande escala também tem descontos por quantidade.\\
$\begin{array}{l l l}
\text{Média} & & \\
\overline{x} & = & \frac{1}{n}\sum_{i=1}^n x_i
\end{array}$
MEMS devices and structures are fabricated using conventional integrated circuit process techniques, such as lithography, deposition, and etching, together with a broad range of specially developed micromachining techniques. \cite{book-9}
The three essential elements in conventional silicon processing are deposition, lithography, and etching. \cite{book-9}
Sensitivity,Long-Term Drift e Temperature Effects (Span temperature hysteresis).
\end{comment}
%%%%%%%%%%%%%%%%%%%%%%%%%%%%%%%%%%%%%%%%%%%%%%%%%%%%%%%%%%%%%%%%

		
%\chapter{Software}
\label{Chapter3}
O \ac{ide} utilizado neste trabalho foi o \textbf{\textit{{Microchip Studio for AVR\textsuperscript{\textregistered} and SAM Devices}}} (\textit{version: 7.0.2542}). A programação foi feita em Linguagem \textbf{C}, sua estrutura sintática esta abaixo mencionado:
\\
\\
\begin{figure}[H]
	\centering
	\includegraphics[scale=0.6]{./image/PESTA/flowchart/Main_Program_1.jpg}
	\caption{Estrutura do Programa (fluxograma)}
	\label{Main_Program_1}
\end{figure}
\figurespace{.5}
O \textit{PROGRAM 1} é onde corre o programa da balança, e o \textit{PROGRAM 2} usado para calibração do \textit{Gain Factor}.
\\
\\
Todos os programas sequem uma estrutura sintática recursiva usando o seguinte modelo.
\\
\\
\begin{figure}[H]
	\centering
	\includegraphics[scale=0.40]{./image/PESTA/flowchart/Generic_structure.jpg}
	\caption{Sintaxe Genérica dos programas (fluxograma)}
	\label{Geneic_structure}
\end{figure}
\figurespace{.5}
Duas interrupções periódicas estão sempre a correr em \textit{background}, uma para fazer o \textit{shift} dos \textit{bit´s} da conversão \textbf{ADC} feita pelo amplificador de sinal HX711 e outra interrupção periódica de segundo em segundo usado para saltar de \textit{Menu} pelos botões.
\\
\\
\begin{minipage}{\linewidth}
	\begin{minipage}{.5\linewidth}
		\begin{figure}[H]
			\centering
			\includegraphics[scale=0.7]{./image/PESTA/flowchart/Interrupt_1.jpg}
			\caption{\textbf{ADC} conversão}
			\label{Interrupt_1}
		\end{figure}
	\end{minipage}
	\begin{minipage}{.5\linewidth}
		\begin{figure}[H]
			\centering
			\includegraphics[scale=0.7]{./image/PESTA/flowchart/Interrupt_2.jpg}
			\caption{Saltar de \textit{Menu}}
			\label{Interrupt_2}
		\end{figure}
	\end{minipage}
\end{minipage}
\minipagespace{.5}
Consultar código para leitura das rotinas de interrupção nas folhas \textit{anexas}.
\\
\\
\begin{minipage}{.40\linewidth}
	\begin{figure}[H]
		\flushleft
		\captionsetup{justification=raggedright,singlelinecheck=false}
		\includegraphics[scale=0.9]{./image/PESTA/Code/Livrarias.jpg}
		\caption{Livrarias}
		\label{Livrarias}
	\end{figure}
\end{minipage}
\begin{minipage}{.6\linewidth}
	Ao lado esta as livrarias usadas neste projeto, as que estão dentro da caixa amarela são as que foram criadas.
	A filosofia usada é de criar objetos que representam o hardware para o poder manipular via código. Como se pode observar foi criado uma livraria para os temporizadores, outra para as interrupções e \ac{eeprom}, depois criado livrarias para os componentes externos, isto é, o \textbf{LCD} e o integrado \textbf{HX711}. \\
	Uma abstração que torna simples executar qualquer algoritmo ou projeto, e isto só é possível depois de ultrapassar a barreira árdua e dolorosa de desenvolver as livrarias.
	\\
	\\
	\\
\end{minipage}
\minipagespace{.5}
Durante este projeto tentou-se dentro do possível sempre seguir as boas praticas de programação, como a \texttt{indentação} especifica para cada situação, e manter a estrutura de norma. Deve-se manter uma metodologia de trabalho que segue as normas assim é percetível para todos e facilita detetar \textit{bugs}, e é uma pratica que abrange todas as linguagens, por exemplo o \textbf{Python} é fundado nesse principio.
\\
\\
Quanto as interrupções é sempre um desafio, porque as tarefas tem que ser bem organizadas temporalmente para não entrar em conflitos, pretende-se assim sempre que o código corra na função \textit{main} e as interrupções algo esporádico e muito rápido, estas devem servir de \textit{flags} para executar rotinas na função \textit{main}, ou seja, servirem de \textbf{\textcolor{green}{INPUTS}}.
\newpage
\section{Validação}
%%%To validate is to justify why the choices made and alternatives that could be chosen.
As escolhas feitas estão dentro dos parâmetros da oferta disponibilizada. Apenas o conhecimento adquirido ao aprofundar o funcionamento dos componentes é o ganho mais evidente, facilitando a interpretação de situações e deteção de anomalias (\textit{troubleshooting}), derivado aos custos dos materiais serem caros.\\
\\
Apostei na marca \textbf{Atmel} devido a experiência e conhecimentos já adquiridos, se aposta-se noutra marca teria de enfrentar uma curva de aprendizagem e adaptação que no final a nível de custos beneficio seria desfavorável, pelo tempo a dispensar e de ser muito trabalhoso a refazer tudo novamente noutra arquitetura.\\
\\
O sensor usado é o mais comum nesta pratica, e escolha demonstrada, o circuito de interface é indiferente a escolha apenas é baseada na sua precisão, ou seja, é de \textcolor{blue}{24} \textit{bit} enquanto o \textbf{ADC} do \textbf{MCU} de \textcolor{blue}{10} \textit{bit}.
\\
\\
\\
\subsection{Material}
Abaixo esta indicado uma tabela dos materiais usados, assim como os preços.
\\
\\
\begin{table}[H]{
		\caption{Lista de material}
		\rowcolors{3}{blue!80!yellow!50}{blue!70!yellow!40}
		\begin{tabular}{ |p{12cm}|c|p{2cm}|  }
			\hline
			\multicolumn{3}{|c|}{Lista de Material} \\
			\hline
			Peça & Quant & Preço [uni] \\
			\hline
			Fonte de alimetação 12V 1A & 1 & \EUR{3.87} \\
			Conversor DC-DC com voltímetro & 1 & \EUR{7.75} \\
			ET BASE AVR Atmega128 Board & 1 & \EUR{23.92} \\
			Test Input Board  & 1 & \EUR{3.71} \\
			Test Output Board & 1 & \EUR{3.71} \\
			IDC Socket 10 way    & 12 & \EUR{0.31} \\
			IDC Header Straight 10 way    & 12 & \EUR{0.25} \\
			Flatcable    & ? & \EUR{?} \\
			20x4 LCD Module Blue & 1 & \EUR{12.24} \\
			SparkFun Load Cell Amplifier HX711 & 1 & \EUR{13.04}   \\
			50Kg Load Cell & 1 & \EUR{12} \\
			\hline
			& \textit{total} & \EUR{86.96} \\
			\hline
		\end{tabular}
	}
	\label{material}
\end{table}
\newpage
\subsection{Testar}
Quanto a funcionalidade no seu todo a balança tem \textcolor{blue}{quatro} botões e \textcolor{blue}{três} \textit{leds} ativados, um botão para fazer o \textit{offset} no \textcolor{green}{PORTF 0}, e dois botões com dupla função, fazer \textit{reset} para \textit{default} e incrementar, outro para entrar no menu de calibração e decrementar, o \textcolor{blue}{quarto} botão é reservado para \textit{enter} e assumir o valor introduzido na calibração.\\
\\
O botão \textcolor{green}{PORTF 3} quando premido durante \textcolor{blue}{cinco} segundos faz um \textit{reset} para configuração \textit{default} depois de o \textit{led} no \textcolor{red}{PORTC 6} piscar \textcolor{blue}{quatro} vezes.\\
\\
O botão \textcolor{green}{PORTF 4} quando premido durante \textcolor{blue}{cinco} segundos entra no menu de calibração do valor do \textit{gain factor} e o \textit{led} no \textcolor{red}{PORTC 7} liga, usando os botões de incrementa e decrementar, isto é, o
\textcolor{green}{PORTF 3} e \textcolor{green}{PORTF 4} pode-se alterar esse valor.\\
\\
Para assumir o valor e sair do menu de calibração basta premir o botão colocado no \textcolor{green}{PORTF 5}. Tanto no caso de calibração ou de \textit{offset} os valores são guardados na \textbf{EEPROM} do microcontrolador, sendo que, se retirar a alimentação do circuito este não perde os valores e o \textit{led} \textcolor{red}{PORTC 5} permanece ligado.
\\
\\
%%%%%%%%%%%%%%%%%%%%%%%%%%%%%%%%%%%%%%%%%%%%%%%%%%%%%%%%%%%%%%%%
\begin{comment}
Sem contar com as despesas no equipamento para a programação do hardware que em principio só se gasta uma vez, isto é, se não se estragar. No caso do programador \textbf{Atmel-ICE} pode custar até \EUR{185.55}.\\
\\
É de ter em conta que os preços são \textbf{PVP}, que no caso se for preços comerciais são dez vezes inferior, e se for para produção em grande escala também tem descontos por quantidade.\\
$\begin{array}{l l l}
\text{Média} & & \\
\overline{x} & = & \frac{1}{n}\sum_{i=1}^n x_i
\end{array}$
MEMS devices and structures are fabricated using conventional integrated circuit process techniques, such as lithography, deposition, and etching, together with a broad range of specially developed micromachining techniques. \cite{book-9}
The three essential elements in conventional silicon processing are deposition, lithography, and etching. \cite{book-9}
Sensitivity,Long-Term Drift e Temperature Effects (Span temperature hysteresis).
\end{comment}
%%%%%%%%%%%%%%%%%%%%%%%%%%%%%%%%%%%%%%%%%%%%%%%%%%%%%%%%%%%%%%%%
	% Uncomment the lines as you write the chapters,
								% where 'chapter3' refers to a file chapter3.tex
%...							% in the 'chapters' folder
								
%%%%%%%%%%%%%%%%%%%%%%%%%%%%%%%%%%%% Chapter Template

\chapter{Conclusões} 	% Main chapter title
\label{Chapter6} 		% For referencing the chapter elsewhere, usage \ref{Chapter6}

%%%%%%%%%%%%%%%%%%%%%%%%%%%%%%%%%%%%

\lipsum[1]

%%%%%%%%%%%%%%%%%%%%%%%%%%%%%%%%%%%%

\section{Trabalho Futuro}
\label{sec:Ch6.1}

\lipsum[1] 







%%%%%%%%%%%%%%%%%%%%%%%%%%%%%%%%%%%%%%%%%%%%%%%%%%%%%%%%%%%%%%%%%%%%%%%%%%%%%%%%%%%%%%%%%
% Print the bibliographic references using the ieeetr format from the 'sampleRefs.bib' file (root folder)

\printrefereces{sampleRefs}		% Change the 'sampleRefs' name to match your .bib file name,
								% a good option to make bib files is https://www.jabref.org/
%%%%%%%%%%%%%%%%%%%%%%%%%%%%%%%%%%%%%%%%%%%%%%%%%%%%%%%%%%%%%%%%%%%%%%%%%%%%%%%%%%%%%%%%%
\appendix
% Include the appendices of the document as separate files from the 'chapters' folder

% Appendix A

\chapter{Título do Anexo} % Main appendix title
\label{AppendixA} % For referencing this appendix elsewhere, use \ref{AppendixA}

%%%%%%%%%%%%%%%%%%%%%%%%%%%%%%%%%%%%%%%%%%%%%%%%%%%%%%%%%%%%%%%%%%%%%%%%%%%%%%%%%%
\section{Secção}

\lipsum[1]

%%%%%%%%%%%%%%%%%%%%%%%%%%%%%%%%%%%%%%%%%%%%%%%%%%%%%%%%%%%%%%%%%%%%%%%%%%%%%%%%%%
\section{Mais uma secção do Anexo~\ref{AppendixA}}

\lipsum[1]

	% Uncomment the lines as you write the appendices,
%\include{chapters/appendixB}	% or comment the lines if not used

%%%%%%%%%%%%%%%%%%%%%%%%%%%%%%%%%%%%%%%%%%%%%%%%%%%%%%%%%%%%%%%%%%%%%%%%%%%%%%%%%%%%%%%%%
\end{document}  
