\chapter{Introdução}
Num mundo altamente mecanizado, a força e o torque dentro de todas as grandezas, são as mais comuns. Estas têm um papel significativo nos aparelhos de medição de massa e células de carga usadas na indústria e retalho, nos automóveis e nas aeronaves, no aperto de tampas de frascos de medicina e parafusos.\cite{book-9}\\
\\
A massa é uma das grandezas fundamentais, uma propriedade intrínseca de um objeto, que se mede pela sua resistência à aceleração.\cite{book-2}
%melhorar imagen
\begin{figure}[H]
	\centering
	\includegraphics[width=\linewidth]{./image/PESTA/fisica/mass.jpg}
	\caption{Peso e Massa}
	\label{mass}
\end{figure}
Através da medição do peso, isto é, da força gravitacional exercida num dado objeto, podemos calcular a massa. \cite{book-2}
\\
\\
A massa dos objetos pode ser medida por balanças utilizando vários métodos, tais como as alavancas convencionais, por  molas e sistemas mais modernos, isto é, balanças digitais.
\\
\\
As balanças digitais estão presentes no nosso dia a dia, e tornaram-se numa ferramenta indispensável na indústria, comércio e laboratórios. Determinar a massa dos objetos facilitou a moeda de troca no comércio, e tornou-se num pilar fundamental da física e seu desenvolvimentos. Existe uma demanda, tanto na área comercial como na industrial, que empurra o desenvolvimento dos sensores para obter melhores resultados quanto a precisão e imunidade de influências exteriores na medição das grandezas mencionadas.\\
\\
A importância desta grandeza (massa), foi o mote para o projeto proposto: a implementação de uma balança digital para uso doméstico, usando os meios que estão disponíveis no mercado.
\newpage
\section{Objetivos}
O objetivo principal deste projeto será a implementação de uma balança digital economicamente viável, usando os sensores e equipamentos disponíveis no mercado, para ter um produto útil e fácil de ser replicado.
\\
\\
Escolher o sensor adequado e meios de tratamento da informação e comunicação, será o objeto de estudo. Tendo como foco a utilização de um \textit{Embeded System} utilizando as ferramentas necessárias para o executar.
\\
\\
No final o produto obtido será testado, de forma a tentar aperfeiçoar, fazer melhorias, alterações e adaptações que possam surgir, de forma a simplificar o projeto e torná-lo mais atraente ao consumidor final.
\newpage
\section{Calendarização}
O segundo semestre teve inicio em 8 de Março, num ambiente de pandemia COVID-19 período durante o qual fomos forçados ao ensino à distancia, e todo o trabalho teve de ser acompanhado \textit{online} pelos docentes. No entanto foi necessário organizar as tarefas pretendidas de acordo com o plano abaixo descrito, para poder fazer a entrega deste trabalho antes da data limite da época normal (28 de Junho de 2021).
\\
\\
%%%%%%%%%%%%%%%%%%%%%%%%%%%%%%%%%%%%%%%%%%%%%%%%%%%%%%%%%%%%%%%%
\begin{table}[H]
	\caption{Calendarização das tarefas}
	%\begin{sidewaysfigure}
	\begin{ganttchart}[vgrid, hgrid]{1}{20}
		\gantttitle{Março}{5} 
		\gantttitle{Abril}{5}
		\gantttitle{Maio}{5}
		\gantttitle{Juno}{5}\\
		\gantttitlelist{1,...,20}{1}\\
		%First Group
		\ganttgroup{Requisitos}{2}{10} \\
		\ganttbar{Material}{3}{5} \\
		\ganttbar{\textit{Template} LaTeX}{5}{10} \\
		\ganttbar{\textbf{IDE} \textit{Template}}{3}{10}\\
		%\ganttlink{elem0}{elem1}
		%\ganttlink{elem1}{elem2}
		%\ganttlink{elem2}{elem3}
		%\ganttmilestone{Milestone 1}{11}
		%Second Group
		\ganttgroup{Projecto}{3}{20} \\
		\ganttbar{Kit Desenvolvimento}{3}{4} \\
		\ganttbar{Montagen Mesa Sensor}{5}{7} \\%5 7
		\ganttbar{HX711 comunicação}{4}{10} \\%4 8
		\ganttbar{Programação e Ensaio}{4}{20}\\
		%\ganttlink{elem4}{elem5}
		%\ganttlink{elem5}{elem6}
		%\ganttlink{elem6}{elem7}
		%\ganttmilestone{Milestone 1}{11}
		%Third Group
		\ganttgroup{Relatório}{9}{20} \\
		\ganttbar{Literartura}{8}{12} \\
		\ganttbar{Análise documentação}{9}{12} \\
		%\ganttbar{Validação}{10}{12} \\
		\ganttbar{Execução}{10}{20}
		%\ganttlink{elem8}{elem9}
		%\ganttlink{elem9}{elem10}
		%\ganttlink{elem10}{elem11}
		%\ganttmilestone{Milestone 1}{11}
	\end{ganttchart}
	\label{gantt}
%\end{sidewaysfigure}
\end{table}
\newpage
\section{Organização do Relatório}
No capítulo 1 é feita uma contextualização do trabalho realizado e do seu propósito face às necessidades do mundo atual.
\\
\\
No capítulo 2 é apresentada uma breve história da evolução das balanças, e depois um estudo da balança digital.
\\
\\
No capitulo 3 é apresentado a balança digital, seus componentes e características descrevendo seus funcionamentos.
\\
\\
No capitulo 4 é descrito o software utilizado, estrutura do programa, e a implementação do código.
\\
\\
O capitulo 5 são exposto as conclusões e futuras possíveis alterações de melhoramento do produto.

%%%%%%%%%%%%%%%%%%%%%%%%%%%%%%%%%%%%%%%%%%%%%%%%%%%%%%%%%%%%%%%%
\chapter{Evolução da balança}
%%%%%%%%%%%%%%%%%%%%%%%%%%%%%%%%%%%%%%%%%%%%%%%%%%%%%%%%%%%%%%%%
\section{O aparecimento e a evolução da balança}
As balanças foram criadas por necessidade durante o desenvolvimento do comércio na antiguidade. Os produtos que não recorriam a contagem por unidades, tais como objetos irregulares (por exemplo o ouro) tinha de se quantificar o seu valor, e a forma de medir a sua massa tornou-se numa variável de medição para a troca de bens.
\\
\\
A relíquia mais antiga de uma balança de medir massa foi descoberta na vila de \textit{Indus River}, perto da região hoje conhecida como Paquistão, e estima-se ser por volta de \textcolor{blue}{2000} A.C.
\\
Estas primeiras balanças eram alavancas em equilíbrio. $[ \; F_{1} \times b_{1c} = F_{2} \times b_{2c} \; ]$. Nos extremos eram colocados cestos estando este conjunto centrado no seu centro de massa. Assim, se os pesos colocados nos dois cestos fossem iguais, a alavanca ficava em equilíbrio (na horizontal). Esse tipo de balança era um sistema de comparação, com recurso a pesos fixos estabelecidos como norma e designados de \textbf{contra-pesos}.
\\
\begin{minipage}[!b]{0.45\linewidth}
	\begin{figure}[H]
		\centering
		\includegraphics[height=7cm]{./image/PESTA/general/balanca_1.jpg}
		\caption{Balança medieval}
		\label{balanca_1}
	\end{figure}
\end{minipage}
\hspace{2.2cm}
\begin{minipage}[!b]{0.45\linewidth}
	\begin{figure}[H]
		\centering
		\includegraphics[height=7cm]{./image/PESTA/general/balanca_4.jpg}
		\caption{Balança de laboratório}
		%\caption{Balança moderna \cite{book-7}}
		\label{balanca_4}
	\end{figure}
\end{minipage}
\newline
\newline
\newline
Este sistema tem grande precisão, mas também pode se facilmente ser adulterado.
\newpage
Os métodos de medir a massa de objetos não conheceram nenhumas melhorias tecnológicas relevantes até à era industrial. Só nos fins do século \textcolor{blue}{\textit{XVIII}} é que o meio de medir a massa de objetos não dependia de \textbf{contra-pesos}. As balanças por molas foram inventados por volta de \textcolor{blue}{1770} em Inglaterra por \textit{Richard Salter}, um fabricante de balanças.
\\
\begin{figure}[H]
	\centering
	\includegraphics[scale=.3]{./image/PESTA/general/Weigh_Scale_Salter_1.jpg}
	\caption{Balança de Salter}
	\url{https://en.wikipedia.org/wiki/Salter_Housewares}
	\label{Weigh_Scale_Salter_1}
\end{figure}
A balança por mola, como o nome implica, mede a pressão (ou sua tensão) exercida sobre a mola para determinar a massa do objeto. Este tipo de balanças ainda é muito comum nos dias de hoje, por serem bastante económicas de fabricar, mas não têm tanta precisão como as eletrónicas desenvolvidas e aperfeiçoadas durante o século \textcolor{blue}{\textit{XX}}.
\\
\begin{minipage}[!b]{\linewidth}
	\begin{figure}[H]
		\captionsetup{justification=raggedright,singlelinecheck=false}
		\flushleft
		\includegraphics[height=7cm]{./image/PESTA/general/Public_Body_Scales_1.jpg}
		\hspace{.8cm}
		\includegraphics[height=7cm]{./image/PESTA/general/Balanca_Mola_1.jpg}
		\caption{Balanças de Mola}
		\label{Balanca_Mola_1}
	\end{figure}
\end{minipage}
\minipagespace{.5}
\section{A balança eletrónica}
%Desenvolver mais
As balanças eletrónicas mais modernas, utilizam resistências elétricas instaladas sobre materiais flexíveis por onde passa uma corrente elétrica, na qual é possível detetar a variação de condutividade das resistências. Esta variação é proporcional à pressão exercida sobre esse material, podendo dai obter-se o peso dos objetos.
\\
\begin{figure}[H]
	\centering
	\includegraphics[height=7cm]{./image/PESTA/general/Scale_1.jpg}
	\caption{Balança eletrónica}
	\label{Scale_1}
\end{figure}
No projeto proposto será utilizada uma \textbf{célula de carga}, que segue o princípio acima mencionado. Estas células têm sensores \textit{\textbf{strain gauges}} ligadas em ponte de \textit{Wheatstone}, que vão detetar a distorção (pressão) do material, ou seja, da célula de carga e gerar um sinal de diferença de potencial proporcional à força exercida. Seque o mesmo princípio de uma mola.
%As expressões aparecem sem ligação ao texto
%Deve desenvolver mais esta parte, resultando numa mais elaborada fundamentação teórica
\begin{equation}
	\label{eq:Hooke}
	K = \frac{\Delta l}{F}
\end{equation}
Outros tipos de células de carga, tais como as pneumáticas e hidráulicas, convertem a pressão num sinal elétrico, que é proporcional à força nela exercida, de acordo com a expressão \ref{eq:Preasure}.
\begin{equation}
	\label{eq:Preasure}
	P = \frac{F}{A}
\end{equation}
As células de carga capacitivas são outro exemplo de como obter um sinal proporcional da força imposta como carga. Neste caso, é medida a sua capacidade de acordo com o afastamento ou aproximação das placas dos elétrodos.
\begin{equation}
	\label{eq:Capacity}
	C = \varepsilon_{0} \; \varepsilon_{r} \; \frac{A}{d}
\end{equation}
Também existem células, que utilizam o princípio de ressonância, desfasamento de fase, e pelo efeito \textit{Doppler} para determinar a pressão ou distorção das células de carga e o que resulta numa medição.
\\
\\
Pode-se dizer que, em todos os casos se determina a força resultante através do deslocamento no espaço.
%%%%%%%%%%%%%%%%%%%%%%%%%%%%%%%%%%%%%%%%%%%%%%%%%%%%%%%%%%%%%%%%
\begin{comment}
Measurement devices need to be robust to withstand changing environmental influences such as temperature, vibration, and humidity, and they must also provide reliable measurement over long periods of time. Mechanical interfacing of the devices can be difficult and can influence final measurement. The forces and torques may change rapidly, and so the devices must have adequate frequency and transient responses.\\
There are several methods to measure forces and torques. Often, the force to be measured is converted into a change in length of a spring element. The change in dimensions is subsequently measured by a sensor, for example, a piezoresistive, a capacitive or a resonant sensor.\\
It is not so surprising, therefore, that most force and torque measurement devices utilize the long and well-established resistance strain gauge technology.\\
Unfortunately, the metallic resistance strain gauge is relatively insensitive such that in use it is normal to obtain only several millivolts of analog voltage before amplification, and the gauges must not be significantly overstrained. The rangeability and overloading capabilities are seriously restricted. Also, the gauges consume relatively high electrical power (e.g., 250 mW).\\
In general, measurement instrumentation now needs smaller sensing devices of lower power consumption and with greater rangeability and overload capabilities.\\
Greater compatibility with digital microelectronics is highly desirable. Noncontact and wireless operation is sometimes needed, and in some cases batteryless devices are desirable. Production of measurement devices using metallic resistance strain gauges can be relatively labor intensive and skilled, and may require relatively ineffi-cient calibration procedures.\\
In recent years some instrument manufacturers of force and torque measure-ment devices have moved away from using resistance strain gauges. Already, one leading manufacturer of weighing machines for retail and industrial applications now uses metallic and quartz resonant tuning fork technologies, and smaller compa-nies have established niche markets using surface acoustic wave (SAW) technology, optical technology, and magnetoelastic technology.\\
Further commercial developments are taking place to enhance device manufacturability and improve device sensitivity and robustness in operation. Measurement on stiffer structures at much lower strain levels is now possible. The worldwide sensor research base is very active in exploring MEMS for sensing force and torque, and the rest of this chapter will review the current situation and future prospects.\\
\\
The market pull provided by the automotive industry—for example, for manifold air pressure sensors has led to the development of successful devices and technologies that have benefited a wide range of other pressure sensing applications.
\end{comment}
