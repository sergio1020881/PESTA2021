\chapter{Balança}
%\setcounter{section}{0}
%%%%%%%%%%%%%%%%%%%%%%%%%%%%%%%%%%%%%%%%%%%%%%%%%%%%%%%%%%%%%%%%
As balanças foram criadas por necessidade, quando o desenvolvimento de comercio durante a antiguidade os produtos que não recorriam a contagem por unidades, tais como objetos irregulares por exemplo o ouro, a forma de medir sua massas tornou-se numa variável de medição para troca de bens.\\
\\
A relíquia mais antiga de uma balança de medir massa foi descoberto na vila de \textit{Indus River}, perto do conhecido por hoje de Pakistão, e estima-se ser por volta de 2000 B.C.\\
Estas primeiras balanças eram balanças de equilíbrio, tendo um braço onde nos extremos eram colocados cestos e se colocava os pesos, este estava centrado no seu centro de massa, assim se os pesos nos dois cestos fossem iguais ficava paralelo ao solo, era um sistema de comparar com pesos fixos estabelecidos como norma.
\\
\begin{figure}[H]
	\centering
	\includegraphics[scale=0.52]{./image/PESTA/general/balanca_1.jpg}
	\caption{balança medieval}
	\label{Balanca_1}
\end{figure}
Este sistema pode ter grande precisão, mas também pode facilmente ser adulterado.
\\
\\
\\
The weighing scale didn’t know any major technological improvements until the industrial era. It is only starting in the late 18th century that new ways to measure mass appeared that didn’t rely on counter-weights. The spring scale was invented by Richard Salter, a British balance maker around 1770. The spring scale, as the name implies, measures the pressure (or the tension) exerted on a spring to deduce the weight of an object. Spring scales are still fairly common today because they are very cheap to make, but they are not quite as accurate as the electronic systems designed and perfected during the 20th century.
\\
\\
\\

The most modern body scales rely on electronics to measure the weight of their users. By sticking electrical resistances on deformable materials and running a current through them, it is possible to detect variations in the conductivity of the resistances that are correlated to the amount of pressure exerted on the material, and thus to deduce the weight of the person (or the object) standing on the scale. The most high-end body scales also act as impedance meters, and are able to calculate the ratio of fat mass and lean mass in the body. The impedance measurement is taken by generating a very small electrical current on the surface of the scale and measuring the resistance encountered by the current as it travels through the body. Lean mass is a better conductor than fat mass, so it is therefore possible to deduce the ratio of both in the body.
\\
\\
\\

This groundbreaking find calls into question the assumption that early Levantine settlements were less technologically or economically developed than those found in present-day Turkey and Greece. In fact, the scale beam dates back to the early third millennium BC , predating those discovered elsewhere, while the location of the find at Tell Fadous-Kfraabida–believed to be a secondary Bronze Age urban settlement–may indicate that the technology was already widespread in the region at the time.

\section{section}
\subsection{subsection}
\subsection{subsection}
\section{subsection}
