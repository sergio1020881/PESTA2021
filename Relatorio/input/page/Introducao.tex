\chapter{Balança}
%\setcounter{section}{0}
%%%%%%%%%%%%%%%%%%%%%%%%%%%%%%%%%%%%%%%%%%%%%%%%%%%%%%%%%%%%%%%%
As balanças foram criadas por necessidade, quando o desenvolvimento de comercio durante a antiguidade os produtos que não recorriam a contagem por unidades, tais como objetos irregulares por exemplo o ouro, a forma de medir sua massas tornou-se numa variável de medição para troca de bens.\\
\\
A relíquia mais antiga de uma balança de medir massa foi descoberto na vila de \textit{Indus River}, perto do conhecido por hoje de Pakistão, e estima-se ser por volta de 2000 B.C.\\
Estas primeiras balanças eram balanças de equilíbrio, tendo um braço onde nos extremos eram colocados cestos e se colocava os pesos, este estava centrado no seu centro de massa, assim se os pesos nos dois cestos fossem iguais ficava paralelo ao solo, era um sistema de comparar com pesos fixos estabelecidos como norma (\textit{contra-pesos}).
\\
\begin{figure}[H]
	\centering
	\includegraphics[scale=0.52]{./image/PESTA/general/balanca_1.jpg}
	\caption{Balança medieval}
	\label{Balanca_1}
\end{figure}
Este sistema pode ter grande precisão, mas também pode facilmente ser adulterado.
\\
\\
Os métodos de medir a massa de objetos não conheceu nenhumas melhorias tecnológicas relevantes até a era industrial. Só nos fins do século \textit{XVIII} é que o meio de medir a massa de objetos não dependia de \textbf{contra-pesos}. As balanças por molas foi inventado por \textbf{\textit{Richard Salter}}, um fabricante de balanças por volta dos anos de 1770 na Inglaterra.\\
\begin{figure}[H]
	\centering
	\includegraphics[scale=0.9]{./image/PESTA/general/public_body_scales_1.jpg}
	\caption{Public Body Scale}
	\label{public_body_scale_1}
\end{figure}
A balança por mola, como o nome implica, mede a pressão (ou sua tensão) exercido sobre a mola para determinar a massa do objeto. Este tipo de balanças ainda são muito comum nos dias de hoje por serem bastante económicas de fabricar, mas não tem tanta precisão como as eletrónicas desenvolvidas e aperfeiçoadas durante o século \textit{XX}.
\\
\\
The most modern body scales rely on electronics to measure the weight of their users. By sticking electrical resistances on deformable materials and running a current through them, it is possible to detect variations in the conductivity of the resistances that are correlated to the amount of pressure exerted on the material, and thus to deduce the weight of the person (or the object) standing on the scale.
\\
\\
\\
\section{section}
\subsection{subsection}
\subsection{subsection}
\section{subsection}
