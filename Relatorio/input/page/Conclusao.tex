\chapter{Conclusão}
Ao longo deste texto foram sendo apresentadas conclusões que permitiram sustentar as opções de desenvolvimento efetuadas ao longo do projeto. Assim, nesta última secção é realizada uma síntese das principais conclusões, consequências e relevância do trabalho realizado e perspetivados futuros desenvolvimentos.
\\
\\
Sob pena de repetição, é inevitável voltar a recordar a condicionante que acabou por determinar a metodologia de desenvolvimento adotada. O constrangimento encontrado no desenvolvimento da solução de recolha de Alarmes através da interface \textit{Java Messaging Service} (JMS), além de ter resultado num atraso inesperado no desenvolvimento do projeto, impediu explorar as capacidades de correlação dos alarmes, que se propagam por diversos Elementos de Rede (ER), e a determinação da root-cause (que assume um papel da máxima importância).
\\
\\
A tradução das \textit{traps} pelo módulo \textit{snmptrapd} revelou-se de grande utilidade, pois contribuiu para o aumento da produtividade da equipa de desenvolvimento e tem permitido o desenvolvimento de outros coletores de uma forma mais simples e célere.
\\
\\
Em termos funcionais as aplicações desenvolvidas foram verificadas face aos objetivos pretendidos e validadas pelos futuros utilizadores. No que respeita à avaliação do desempenho, a fase de teste correu dentro do esperado, com a manipulação de uma média (próxima) de 16 000 alarmes por dia, que geram uma média de 5 000 notificações por dia.
%%%%%%%%%%%%%%%%%%%%%%%%%%%%%%%%%%%%%%%%%%%%%%%%%%%%%%%%%%%%%%%%
\section{aspectos}
%%%%%%%%%%%%%%%%%%%%%%%%%%%%%%%%%%%%%%%%%%%%%%%%%%%%%%%%%%%%%%%%
