\chapter{Conclusão}
%\setcounter{section}{0}
%%%%%%%%%%%%%%%%%%%%%%%%%%%%%%%%%%%%%%%%%%%%%%%%%%%%%%%%%%%%%%%%
\qquad O futuro do trabalho passa por adquirir novas competências, e conseguir adaptação as novas tendências, a procura de oportunidades e valorização pessoal uma mais valia. Já foi demonstrado que é importante desenvolvermos tanto as nossas metodologias de trabalho como a experiência para estarmos preparados para enfrentar os desafios que possam surgir. \\
\\
Ninguém sabe o futuro, muito menos o imprevistos, cabe as novas gerações decidir, a liberdade acho que é algo que todos desejam, a estabilidade e segurança, mas não só depende de nós, o planeta terra, a galáxia e o universo tem palavra soberana.\\
\\
Considero a humildade e gratidão um atributo fundamental, saber que estamos sujeitos a forças maiores e respeitar essas fronteiras, historicamente isso foi comprovado vezes sem conta, e hoje é outra prova disso, como a pandemia.\\
\\
Gerações após gerações existe uma concentração tremenda muito focada com uma visão míope, um convite para o desastre. Sabe-se lá se temos cura. No entanto temos que ter esperança que haja visão e iluminação.
Esta conversa até parece religiosa, mas nada disso.\\
\\
Ainda muitas conclusões pode-se tirar acerca do futuro do trabalho e marketing pessoal, que não entra nos parâmetros deste relatório e discussão, o desenvolvimento é uma forma de enriquecimento mais rápido e eficaz, com contornos sociais complicados, da a entender que vai haver um excesso de produção sem clientes á vista, um problema de reciclagem e sobrevivência, talvez devia haver uma preocupação na  regulamentação.\\
\\
O ensino talvez esta a ficar desadequado para nossos tempos, não incluindo as ferramentas necessárias na formação e treino, o tempo de retenção dos estudantes excessivo, que provoca colisões entre gerações, especialmente num país pequeno com uma cultura estática, sendo quase impossível obter os resultados esperados caindo na decadência social e económica no seu geral, etc, etc, etc.\\
\\
No meu ver \textcolor{green}{Portugal} deixou de ser um país mas uma fábrica, as competências uma forma de sobreviver no sentido de estar numa escala mais favorável, dai que a solução de mudar para uma sociedade onde o valor individual depende das competências origina a juventude fugir deste sistema instalado.\\
\\
Abaixo link de apanhado de dados utilizados para este relatório:\\ \textcolor{green}{\small [ https://padlet.com/sergio1020881/xih48fe75koesxdg ]}
