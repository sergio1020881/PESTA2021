\chapter{Conclusão}
Uma conclusão óbvia é que a humanidade não tem esperança, demorou desde 2000 B.C até 1770 A.D, 3770 anos a descobrir de que se podia deduzir a massa dos objetos através da Lei de \textbf{Hooke}, acho uma desgraça.
\\
\\
Pode-se também inferir que não compensa desenvolver projetos que se basearam no preço de materiais \textbf{PVP}, todo o lucro vai para a área comercial na qual não tem esforço intelectual ou laboral, no entanto se for com finalidade de produção em grande escala e entrar no mercado auferindo das regalias que ela oferece, tais como o preço da matéria prima talvez compensa se não for excessivamente eliminado pelos gastos de burocracia e estado. \\
\\
\\
Importância dos equipamentos ou ferramentas usadas no projeto tais como o multímetro e osciloscópio, que nos facilita em afinações e ajustes, dai necessário serem de confiança e fiáveis. Além do já mencionado constantemente a necessidade da habilidade de interpretar \textit{datasheets} e manuais, sendo uma pré-requisito obrigatório que talvez nem é preciso o mencionar.
\\
\\
Deu para notar o efeito \textit{Long-term drift}, como o equipamento esteve ligados quase dois messes reparou-se que existe uma oscilação do \textit{offset} que não ultrapassou 10 gramas, isto é o intervalo de -10 a +10, deve ser devido ao efeito de ruídos e ou temperatura e ou humidades, um fenómeno de histerese.
\\
\\
O projeto esta disponível no \textit{\textbf{GITHUB}} \textit{link}: \url{https://github.com/sergio1020881/PESTA2021/tree/main/SandBox/ATMEGA128/Atmega128}, e possível fazer \textit{download} para quem quiser emular a experiência.




%%%%%%%%%%%%%%%%%%%%%%%%%%%%%%%%%%%%%%%%%%%%%%%%%%%%%%%%%%%%%%%%
%%\section{aspectos}
\begin{comment}
Sensitivity,Long-Term Drift e Temperature Effects (Span temperature hysteresis).
\end{comment}
%%%%%%%%%%%%%%%%%%%%%%%%%%%%%%%%%%%%%%%%%%%%%%%%%%%%%%%%%%%%%%%%
