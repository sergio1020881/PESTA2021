%%%%%%%%%%%%%%%%%%%%%%%%%%%%%%%%%%%%%%%%%%%%%%%%%%%%%%%%%%
\chapter{Conclusões}
%Este parágrafo não se enquadra nas conclusões pretendidas, que se pretende que sejam relativas aos objetivos inicialmente traçados
%Uma conclusão óbvia é que a humanidade não tem esperança, demorou desde 2000 A.C até 1770 A.D, 3770 anos a descobrir de que se podia deduzir a massa dos objetos através da Lei de \textbf{Hooke}.
%Deve-se iniciar as conclusões indicando se os objetivos inicialmente propostos foram atingidos. De seguida é que devem ser apresentadas outras ideias/conclusões pertinentes, retiradas ao longo do desenvolvimento do trabalho.
Pode-se também inferir que não compensa desenvolver projetos que se basearam no preço de materiais \ac{pvp}, todo o lucro vai para a área comercial, no entanto se for com finalidade de produção em grande escala e entrar no mercado auferindo das regalias que ela oferece, tais como o preço da matéria prima, talvez compensa se não for excessivamente eliminado pelos gastos de burocracia e estado.
%reescrever este parágrafo, desenvolvendo melhor a ideia que pretende transmitir
\\
\\
A importância dos equipamentos ou ferramentas usadas no projeto, tais como o multímetro, osciloscópio e \acsp{ide}, que nos facilita em afinações e ajustes saltando por cima de muita \textit{red tape}, dai necessário serem de confiança e fiáveis. Além do já mencionado constantemente a necessidade da habilidade de interpretar \textit{datasheets} e manuais, sendo um pré-requisito obrigatório que talvez nem é preciso o mencionar.
É de salientar a importância dos equipamentos e das ferramentas utilizadas no desenvolvimento do projeto, que facilita a realização de afinações e ajustes necessários. A fiabilidade deste equipamento é fundamental para os resultados obtidos.
\\
Outro aspeto e referir é a necessário capacidade do aluno na consulta de \textit{datasheets} e manuais, o que é de facto um pré-requisito para o desenvolvimento do trabalho.
\\
\\
Deu para notar o efeito \textit{Long-term drift}, como o equipamento esteve ligados quase dois meses reparou-se que existe uma oscilação do \textit{offset} que não ultrapassou 20 gramas, isto é o intervalo entre -10 a +10. Tal deve ser devido ao efeito de ruídos e/ou temperatura e/ou humidades. O erro é corrigido ao premir o botão de \textit{offset}.
\\
\\
Atualmente, na implementação de um projeto deste tipo, não compensa desenvolver uns \acs{pcb} próprios \acs{pcb}, devido ao seu valor elevado, em comparação com o valor praticado por uma empresa dedicada a tais serviços. Outro motivo...
%Transmitir melhor a ideia..
Hoje em dia não compensa fazer nossos próprios \ac{pcb}, devido ao preço ser muito baixo se for fabricado por uma empresa dedicada a tais serviços, outro motivo é a disponibilidade  de um vasto leque de circuitos de desenvolvimento disponíveis ao consumidor comum, prontos a ser utilizados.
\newpage
%%%%%%%%%%%%%%%%%%%%%%%%%%%%%%%%%%%%%%%%%%%%%%%%%%%%%%%%%%
\section{Trabalho Futuro}
Considero que foi atingido os objetivos impostos, com possibilidade de no futuro melhorar o projecto, tais como funcionar por bateria e integrar um \textit{sleep mode} de forma a desligar o \textit{display} \acs{lcd} e ficar em \textit{standby} até receber um sinal de \textit{wake up}. Ao todo foi uma boa experiência.
\\
\\
O projeto esta disponível no \textit{\textbf{GITHUB}} \textit{link}: \url{https://github.com/sergio1020881/PESTA2021/tree/main/SandBox/ATMEGA128/Atmega128}, e possível fazer \textit{download} para quem quiser emular a experiência.
\\
\\
Anexado ( \ref{anexo-code} )
vai todo o código utilizado neste projeto.
%%%%%%%%%%%%%%%%%%%%%%%%%%%%%%%%%%%%%%%%%%%%%%%%%%%%%%%%%%%%%%%%
%%\section{aspectos}
\begin{comment}
Sensitivity,Long-Term Drift e Temperature Effects (Span temperature hysteresis).
\end{comment}
%%%%%%%%%%%%%%%%%%%%%%%%%%%%%%%%%%%%%%%%%%%%%%%%%%%%%%%%%%%%%%%%
