\begin{verbatimtab}
/************************************************************************
Title: BALANCA COMERCIAL
Author: Sergio Manuel Santos
<sergio.salazar.santos@gmail.com>
File: $Id: MAIN,v 1.8.2.1 21/02/2021 Exp $
License: GNU General Public License
Software: Atmel Studio 7 (ver 7.0.129)
Hardware: Atmega128 by ETT ET-BASE
    -PORTA LCD
    -PORTF pin 6,7 HX711, pin 0 to 5 Buttons, 
        PIN 0 -> OFFSET, 
        PIN 3 -> DEFAULT 5sec press, and up count for div factor, 
        PIN 4 -> ENTER GAIN FACTOR MENU 5 sec press, and down count for 
        div factor, 
        PIN 5 -> ENTER KEY to validate entered value and put in effect.
        PIN 6 and 7 -> dedicated to comunicate with Signal Amplifier
    -PORTC status indicator leds
        PIN 5 -> Indicate using stored values
        PIN 6 -> Reset to default indicator (blinks four times)
        PIN 7 -> In Calibratio Menu (blinking)
Comment:
    nice
************************************************************************/
#define F_CPU 16000000UL
/*
** library
*/
#include <avr/io.h>
#include <avr/pgmspace.h>
#include <avr/interrupt.h>
#include <util/delay.h>
#include <math.h>
#include <inttypes.h>
#include <stdlib.h>
#include <string.h>
#include "explode.h"
//#include "atmega128interrupt.h"
#include "atmega128timer.h"
#include "function.h"
#include "lcd.h"
#include "hx711.h"
#include "eeprom.h"
/*
** Constant and Macro
*/
#ifndef STATUS_REGISTER
#define STATUS_REGISTER SREG
#define GLOBAL_INTERRUPT_ENABLE 7
#endif
#define ZERO 0
#define ONE 1
#define TRUE 1
#define average_n 24 //64 -> 24
#define blink 8
#define IMASK 0x3F
#define _5sec 5
#define _10sec 10
#define minDIV 1
#define maxDIV 255
/*
** Global File variable
*/
EXPLODE F;
LCD0 lcd0;
TIMER_COUNTER0 timer0;
//INTERRUPT intx;
HX711_calibration HX711_data;
HX711_calibration* HX711_ptr;
const uint8_t sizeblock = sizeof(HX711_calibration);
HX711 hx;
float tmp;
EEPROM eprom;

char result[32];
char Menu = '1'; // Main menu selector
uint8_t counter_1 = ZERO;
uint8_t counter_2 = ZERO;
uint8_t signal = ZERO;
uint8_t count=blink;
uint16_t divfactor;
/*
** Header
*/
void PORTINIT();
/****MAIN****/
int main(void)
{
PORTINIT();
HX711_ptr = &HX711_data; // CALIBRATION DATA BUS
/***INICIALIZE OBJECTS***/
F = EXPLODEenable();
FUNC function = FUNCenable();
lcd0 = LCD0enable(&DDRA,&PINA,&PORTA);
timer0 = TIMER_COUNTER0enable(2,2); //2,2
TIMER_COUNTER1 timer1 = TIMER_COUNTER1enable(4,2); //4,2
hx = HX711enable(&DDRF, &PINF, &PORTF, 6, 7); //6,7
eprom = EEPROMenable();
//intx = INTERRUPTenable();
/******/
float value = 0;
float publish = 0;
uint8_t choice;
// Get default values to buss memory
HX711_data.offset_32 = hx.get_cal(&hx)->offset_32;
HX711_data.offset_64 = hx.get_cal(&hx)->offset_64;
HX711_data.offset_128 = hx.get_cal(&hx)->offset_128;
HX711_data.divfactor_32 = hx.get_cal(&hx)->divfactor_32;
HX711_data.divfactor_64 = hx.get_cal(&hx)->divfactor_64;
HX711_data.divfactor_128 = hx.get_cal(&hx)->divfactor_128;
HX711_data.status = hx.get_cal(&hx)->status;
/***Parameters timers***/
timer0.compoutmode(1); // troubleshooting blinking PORTB 5
/***79 and 8  -> 80 us***/
timer0.compare(60); // 8 -> 79 -> 80 us, fine tunned = 8 -> 60 -> 30.4us
timer0.start(8); // 1 -> 32 us , 8 -> 256 us , 32 64 128 256 1024
// to be used to jump menu for calibration in progress
timer1.compoutmodeA(1); // troubleshooting blinking PORTB 6
timer1.compareA(62800); // Freq = 256 -> 62800 -> 2 s
timer1.start(256);
//intx.set(1,0); // Not necessary, if used move IDC from PORTF to
//PORTD with new config pinage.
// HX711 Gain
hx.set_amplify(&hx, 64); // 32 64 128
choice = hx.get_amplify(&hx);
if(choice == 1)
divfactor = (uint16_t) HX711_data.divfactor_128;
if(choice == 2)
divfactor = (uint16_t) HX711_data.divfactor_32;
if(choice == 3)
divfactor = (uint16_t) HX711_data.divfactor_64;
//Get stored calibration values and put them to effect
eprom.read_block(HX711_ptr, (const void*) ZERO, sizeblock);
if(HX711_ptr->status == 1){
    //Load stored value 
    hx.get_cal(&hx)->offset_32 = HX711_ptr->offset_32;
    hx.get_cal(&hx)->offset_64 = HX711_ptr->offset_64;
    hx.get_cal(&hx)->offset_128 = HX711_ptr->offset_128;
    hx.get_cal(&hx)->divfactor_32 = HX711_ptr->divfactor_32;
    hx.get_cal(&hx)->divfactor_64 = HX711_ptr->divfactor_64;
    hx.get_cal(&hx)->divfactor_128 = HX711_ptr->divfactor_128;
    hx.get_cal(&hx)->status=ZERO;
    PORTC &= ~(ONE << 5); // troubleshooting
}
/*********************************************************/
//lcd0.gotoxy(1,0); // for troubleshooting
//lcd0.string_size(function.ftoa(HX711_data.status, result, ZERO), 13);
//lcd0.string_size(function.ftoa(hx.get_cal(&hx)->offset_64, result, ZERO), 13);
/*********************************************************/
while(TRUE){
	/******PREAMBLE******/
	lcd0.reboot(); //Reboot LCD
	F.boot(&F,PINF); //PORTF INPUT READING
	while(hx.query(&hx)); //Catches falling Edge instance, begins bit shifting.
	/***geting data interval***/
	/************INPUT***********/
	// Jump Menu signal
	if(signal == ONE){ //INPUT FROM INTERRUPT SINALS
		Menu = '2';
		signal = ZERO; // ONE SHOT
		lcd0.clear();
	}
	tmp = hx.raw_average(&hx, average_n);
	// average_n  25 or 50, smaller means faster or more readings
	/****************************/
	switch(Menu){
		/***MENU 1***/
		case '1': // Main Program Menu
		lcd0.gotoxy(0,4); //TITLE
		lcd0.string_size("Weight Scale", 12); //TITLE
		/*********************************************/
		//lcd0.gotoxy(1,0); // for troubleshooting
		//lcd0.string_size(function.ftoa(hx.read_raw(&hx), result, ZERO), 13);
		/*********************************************/
		if((F.hl(&F) & IMASK) & ONE){ // calibrate offset by pressing button 1
			HX711_data.offset_32 = tmp;
			HX711_data.offset_64 = tmp;
			HX711_data.offset_128 = tmp;
			HX711_data.status = ONE;
			eprom.update_block(HX711_ptr, (void*) ZERO, sizeblock);
			hx.get_cal(&hx)->offset_32 = HX711_ptr->offset_32;
			hx.get_cal(&hx)->offset_64 = HX711_ptr->offset_64;
			hx.get_cal(&hx)->offset_128 = HX711_ptr->offset_128;
			hx.get_cal(&hx)->status=ZERO;
			PORTC &= ~(ONE << 5);
		}
		if(choice == 1 || choice == 11)
		value = (tmp - hx.get_cal(&hx)->offset_128) / hx.get_cal(&hx)->divfactor_128;
		//value to be published to LCD
		if(choice == 2 || choice == 21)
		value = (tmp - hx.get_cal(&hx)->offset_32) / hx.get_cal(&hx)->divfactor_32;
		//value to be published to LCD
		if(choice == 3 || choice == 31)
		value = (tmp - hx.get_cal(&hx)->offset_64) / hx.get_cal(&hx)->divfactor_64;
		//value to be published to LCD
		/*********************************************/
		//lcd0.gotoxy(3,0); // for troubleshooting
		//lcd0.string_size(function.ftoa(tmp, result, ZERO), 13);
		//lcd0.string_size(function.ftoa(hx.get_cal(&hx)->divfactor_128, result, ZERO), 13);
		//lcd0.string_size(function.ftoa(hx.get_cal(&hx)->offset_128, result, ZERO), 13);
		/*********************************************/
		if (value > 1000 || value < -1000){
			publish = value / 1000;
			lcd0.gotoxy(2,1);
			lcd0.string_size(function.ftoa(publish, result, 3), 13);
			lcd0.string_size("Kg", 4);
		}else{
			publish = value;
			lcd0.gotoxy(2,1);
			lcd0.string_size(function.ftoa(publish, result, ZERO), 13);
			lcd0.string_size("gram", 4);
		}
		break;
		/***MENU 2***/
		case '2': // MANUAL CALIBRATE DIVFACTOR MENU
		/**/
		lcd0.gotoxy(0,1);
		lcd0.string_size("SETUP GAIN FACTOR",17);
		switch(choice){
			case 1: // case 128
			divfactor=hx.get_cal(&hx)->divfactor_128;
			choice=11;
			break;
			case 11: // case 128
			lcd0.gotoxy(2,9);
			if(F.hl(&F) == (ONE << 3)){
				divfactor++;
				if(divfactor > maxDIV)
				divfactor = maxDIV;
			}
			if(F.hl(&F) == (ONE << 4)){
				divfactor--;
				if(divfactor < minDIV)
				divfactor = minDIV;
			}
			HX711_data.divfactor_128 = divfactor;
			lcd0.string_size(function.ui16toa(divfactor),6);
			break;
			case 2: // case 32
			divfactor=hx.get_cal(&hx)->divfactor_32;
			choice=21;
			break;
			case 21: // case 32
			lcd0.gotoxy(2,9);
			if(F.hl(&F) == (ONE << 3)){
				divfactor++;
				if(divfactor > maxDIV)
				divfactor = maxDIV;
			}
			if(F.hl(&F) == (ONE << 4)){
				divfactor--;
				if(divfactor < minDIV)
				divfactor=minDIV;
			}
			HX711_data.divfactor_32 = divfactor;
			lcd0.string_size(function.ui16toa(divfactor),6);
			break;
			case 3: // case 64
			divfactor=hx.get_cal(&hx)->divfactor_64;
			choice=31;
			break;
			case 31: // case 64
			lcd0.gotoxy(2,9);
			if(F.hl(&F) == (ONE << 3)){
				divfactor++;
				if(divfactor > maxDIV)
				divfactor = maxDIV;
			}
			if(F.hl(&F) == (ONE << 4)){
				divfactor--;
				if(divfactor < minDIV)
				divfactor = minDIV;
			}
			HX711_data.divfactor_64 = divfactor;
			lcd0.string_size(function.ui16toa(divfactor),6);
			break;
			default:
			choice = 3;
			break;
		};
		// Exit and store value
		if((F.ll(&F) & IMASK) == (ONE << 5)){ // Button 6
			HX711_data.status = ONE;
			eprom.update_block(HX711_ptr, (void*) ZERO, sizeblock);
			hx.get_cal(&hx)->divfactor_32=divfactor;
			hx.get_cal(&hx)->divfactor_64=divfactor;
			hx.get_cal(&hx)->divfactor_128=divfactor;
			hx.get_cal(&hx)->status=ZERO;
			PORTC &= ~(ONE << 5); // troubleshooting
			PORTC |= (ONE << 7); // troubleshooting
			counter_2 = ZERO;
			Menu = '1';
			lcd0.clear();
		}
		/**/
		break;
		/********************************************************************/
		default:
		Menu = '1';
		break;
	};
}
}
/*
** procedure and function
*/
void PORTINIT(void)
{
	//Control buttons
	PORTF |= IMASK;
	//troubleshooting output
	DDRC = 0xFF;
	PORTC = 0xFF;
}
/*
** interrupt
*/
ISR(TIMER0_COMP_vect) // 15.4 us intervals
{
	/***Block other interrupts during this procedure***/
	uint8_t Sreg;
	Sreg = STATUS_REGISTER;
	STATUS_REGISTER &= ~(ONE << GLOBAL_INTERRUPT_ENABLE);
	//hx.query(&hx);	
	hx.read_raw(&hx);
	/***enable interrupts again***/
	STATUS_REGISTER = Sreg;
}
ISR(TIMER1_COMPA_vect) // 1 second intervals
{
	/***CLEAR EEPROM OFFSET SEQUENCE START***/
	if((F.ll(&F) & IMASK) == (ONE << 3)) //button 4
	counter_1++;
	else if(counter_1 < _5sec+ONE)
	counter_1=ZERO;
	if(counter_1 > _5sec){
		counter_1 = _5sec+ONE; //lock in place
		PORTC ^= (ONE << 6); // troubleshooting
		count--;
		if(!count){ //led blinks x times
			// Delete eeprom memory ZERO
			eprom.update_block(HX711_Default, (void*) ZERO, sizeblock);
			hx.get_cal(&hx)->offset_32 = HX711_Default->offset_32;
			hx.get_cal(&hx)->offset_64 = HX711_Default->offset_64;
			hx.get_cal(&hx)->offset_128 = HX711_Default->offset_128;
			hx.get_cal(&hx)->divfactor_32 = divfactor = HX711_Default->divfactor_32;
			hx.get_cal(&hx)->divfactor_64 = divfactor = HX711_Default->divfactor_64;
			hx.get_cal(&hx)->divfactor_128 = HX711_Default->divfactor_128;
			hx.get_cal(&hx)->status = HX711_Default->status;
			PORTC |= (ONE << 5); // troubleshooting
			PORTC |= (ONE << 6); // troubleshooting
			counter_1 = ZERO;
			count=blink;
		}
	}
	/***CLEAR EEPROM OFFSET SEQUENCE END***/
	/***CAL DIVFACTOR DEFINE START***/
	if((F.ll(&F) & IMASK) == (ONE << 4)) //button 5
	counter_2++;
	else if(counter_2 < _5sec+ONE)
	counter_2=ZERO; //RESET TIMER
	if(counter_2 > _5sec){
		counter_2 = ZERO; //RESET TIMER
		signal = ONE;
		PORTC &= ~(ONE << 7); // troubleshooting
	}
	/***CAL DIVFACTOR DEFINE END***/
}
/***EOF***/
/**** Comment:
Because 24 bit will have to create a vector pointer of the size of 32
bit, then at the end do a cast to *((int32_t*)ptr).
*************/
\end{verbatimtab}
%%%%%%%%%%%%%%%%%%%%%%%%%%%%%%%%%%%%%%%%%%%%%%%%%%%%%%%%%%%%%%%%