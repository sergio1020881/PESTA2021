%%%%%%%%%%%%%%%%%%%%%%%%%%%%%%%%%%%%%%%%%%%%%%%%%%%%%%%%%%%%%%%%
\definecolor{light-gray}{gray}{0.95}
\definecolor{lightgray}{rgb}{.9,.9,.9}
%%%%%%%%%%%%%%%%%%%%%%%ISEP SETTINGS START%%%%%%%%%%%%%%%%%%%%%%
\lstset{
	language=C,							% choose the language of the code
	%basicstyle=\small\ttfamily\color{black},
	basicstyle=\footnotesize\ttfamily\color{black},
	keywordstyle=\bfseries\color{blue},
	numbers=left,						% where to put the line-numbers   
	numberstyle=\tiny\color{gray},		% the size of the fonts that are used for the line-numbers
	stepnumber=1,						% the step between two line-numbers. If it's 1 each line will be numbered
	numbersep=2pt,						% how far the line-numbers are from the code
	xleftmargin=2em,					% keep line numbers inside the margins		
	frame=tb,							% frame: top and bottom lines
	framexleftmargin=1.5em,				% frame: adjusts left margin because the xleftmargin numbers setting
	float=!htb,							% if float defined, let Latex manage it
	aboveskip=8mm,
	belowskip=4mm,
	backgroundcolor=\color{light-gray},
	commentstyle=\color{red},			% comment style
	keywordstyle=\color{blue},			% keyword style
	showspaces=false,					% show spaces adding particular underscores
	showstringspaces=false,				% underline spaces within strings
	showtabs=false,						% show tabs within strings adding particular underscores
	tabsize=2,							% sets default tabsize to 2 spaces
	captionpos=b,						% sets the caption-position to bottom
	breaklines=true,					% sets automatic line breaking
	breakatwhitespace=false,			% sets if automatic breaks should only happen at whitespace
	escapeinside={\%*}{*)},				% if you want to add a comment within your code
	morekeywords={*,var,template,new},	% if you want to add more keywords to the set
	stringstyle=\color{orange},
	columns=fullflexible
}
%%%%%%%%%%%%%%%%%%%%%%%%%%%%%%%%%%%%%%%%%%%%%%%%%%%%%%%%%%
%\begin{comment}
\DTMnewdatestyle{mydateformat}{
	\renewcommand{\DTMdisplaydate}[4]{
		    %\DTMshortweekdayname{##4},\space
		\DTMmonthname{##2} \nobreakspace de \nobreakspace
		    %\number##3,\space        
		\number##1                  
	}
	\renewcommand{\DTMDisplaydate}{\DTMdisplaydate}%
}
%\end{comment}
%%%%%%%%%%%%%%%%%%%%%%%ISEP SETTINGS END%%%%%%%%%%%%%%%%%%
\begin{comment}
\setlistdepth{12}
\newlist{enumitem}{enumerate}{12}
\setlist[enumitem,1]{label=\roman*)}
\setlist[enumitem,2]{label=\alph*)}
\setlist[enumitem,3]{label=\arabic*)}
\setlist[enumitem,4]{label=(\roman*)}
\setlist[enumitem,5]{label=(\alph*)}
\setlist[enumitem,6]{label=(\arabic*)}
\setlist[enumitem,7]{label=\roman*)}
\setlist[enumitem,8]{label=\alph*)}
\setlist[enumitem,9]{label=\arabic*)}
\setlist[enumitem,10]{label=(\roman*)}
\setlist[enumitem,11]{label=(\alph*)}
\setlist[enumitem,12]{label=(\arabic*)}
\end{comment}
%%%%%%%%%%%%%%%%%%%%%%%%%%%%%%%%%%%%%%%%%%%%%%%%%%%%%%%%%%
\begin{comment}
\renewcommand{\labelitemi}{$\bullet$}
\renewcommand{\labelitemii}{$\cdot$}
\renewcommand{\labelitemiii}{$\diamond$}
\renewcommand{\labelitemiv}{$\ast$}
\end{comment}
%%%%%%%%%%%%%%%%%%%%%%%%%%%%%%%%%%%%%%%%%%%%%%%%%%%%%%%%%%
\begin{comment}
\tikzstyle{RECTANGLE_2} = [rectangle, draw, text width=5em, text centered, rounded corners, minimum height=4em]
\tikzstyle{RECTANGLE_3} = [rectangle, rounded corners, minimum width=3cm, minimum height=1cm,text centered, draw=black, fill=red!80]
\tikzstyle{RECTANGLE_4} = [rectangle, draw, fill=blue!20, text width=3cm, text centered, minimum height=4em]
\tikzstyle{RECTANGLE_5} = [rectangle, minimum width=3cm, minimum height=1cm, text centered, text width=3cm]
\tikzstyle{RECTANGLE_6} = [rectangle, draw, fill=blue!20, text width=5em, text centered, rounded corners, minimum height=4em]
\tikzstyle{RECTANGLE_7} = [rectangle, draw, fill=blue!20, text width=5em, text centered, rounded corners, minimum height=4em]
\tikzstyle{RECTANGLE_8} = [rectangle, draw, align=left, fill=blue!20]
\tikzstyle{RECTANGLE_1} = [rectangle, rounded corners, minimum width=1cm, minimum height=1cm,text centered, draw=black, fill=green!%30]
\tikzstyle{DIAMOND_1} = [diamond, draw, fill=blue!20, text width=4.5em, text badly centered, node distance=4cm, inner sep=0pt]
\tikzstyle{DIAMOND_2} = [diamond, minimum width=3cm, minimum height=1cm, text centered, draw=black, fill=green!30]
\tikzstyle{DIAMOND_3} = [diamond, draw, text width=4.5em, text badly centered, node distance=3cm, inner sep=0pt]
\tikzstyle{DIAMOND_4} = [diamond, draw, fill=blue!20, text width=4.5em, text badly centered, node distance=3cm, inner sep=0pt]
\tikzstyle{DIAMOND_5} = [diamond, draw, fill=blue!20, text width=4.5em, text badly centered, node distance=3cm, inner sep=0pt]
\tikzstyle{DIAMOND_6} = [diamond, draw, fill=blue!20, text width=4.5em, text badly centered, node distance=4cm, inner sep=0pt]
\tikzstyle{DIAMOND_7} = [diamond, draw, align=left, fill=blue!20]
\tikzstyle{ELLIPSE_1} = [draw, ellipse,fill=red!20, node distance=3cm, minimum height=2em]
\tikzstyle{ELLIPSE_2} = [draw, ellipse,fill=red!20, node distance=3cm, minimum height=2em]
\tikzstyle{ELLIPSE} = [draw, ellipse,fill=red!20, node distance=3cm, minimum height=2em]
\tikzstyle{TRAPEZIUM_1} = [trapezium,trapezium left angle=70,trapezium right angle=-70,minimum height=0.6cm, draw, fill=blue!20, text width=4.5em, text badly centered, node distance=3cm, inner sep=0pt]
\tikzstyle{TRAPEZIUM_2} = [trapezium, trapezium left angle=70, trapezium right angle=110, minimum width=3cm, minimum height=1cm, text centered, draw=black, fill=blue!30]
\tikzstyle{TRAPEZIUM_3} = [trapezium,trapezium left angle=70,trapezium right angle=-70,minimum height=0.6cm, draw, fill=blue!20, text width=4.5em, text badly centered, node distance=3cm, inner sep=0pt]
\tikzstyle{ARROW} = [thick,->,>=stealth]
\tikzstyle{LINE} = [draw, -latex']
\tikzstyle{MYLINE} = [draw, ->,  thick, shorten <=4pt, shorten >=4pt]
\tikzstyle{TEXT_1}=[draw,text centered,minimum size=6em,text width=5.25cm,text height=0.34cm]
\tikzstyle{TEXT_2}=[draw,text centered,minimum size=2em,text width=2.75cm,text height=0.34cm]
\tikzstyle{TEXT_3}=[draw,minimum size=2.5em,text centered,text width=3.5cm]
\tikzstyle{TEXT_4}=[draw,minimum size=3em,text centered,text width=6.cm]
\tikzstyle{CIRCLE_1}=[draw,shape=circle,inner sep=2pt,text centered, node distance=3.5cm]
\tikzstyle{CIRCLE_2}=[draw,shape=circle,inner sep=4pt,text centered, node distance=3.cm]
\end{comment}
\begin{comment}
%%%%%%%GANTT 1%%%%%%%
\newganttchartelement*{mymilestone}{
	mymilestone/.style={
		shape=isosceles triangle,
		inner sep=0pt,
		draw=cyan,
		top color=white,
		bottom color=cyan!50
	},
	mymilestone incomplete/.style={
		/pgfgantt/mymilestone,
		draw=yellow,
		bottom color=yellow!50
	},
	mymilestone label font=\slshape,
	mymilestone left shift=0pt,
	mymilestone right shift=0pt
}
\newgantttimeslotformat{stardate}{%
	\def\decomposestardate##1.##2\relax{%
		\def\stardateyear{##1}\def\stardateday{##2}%
	}%
	\decomposestardate#1\relax%
	\pgfcalendardatetojulian{\stardateyear-01-01}{#2}%
	\advance#2 by-1\relax%
	\advance#2 by\stardateday\relax%
}
%%%%%%%GANTT 2%%%%%%%
\end{comment}
%%%%%%%%%%%%%%%%%%%%%%%%%%%%%%%%%%%%%%%%%%%%%%%%%%%%%%%%%%
%%%%%%%%%%%%%%%%%%%%%%%%%%%%%%%%%%%%%%%%%%%%%%%%%%%%%%%%%%
\addto{\captionsportuguese}{\renewcommand*\contentsname{\'{I}ndice}}
\addto{\captionsportuguese}{\renewcommand*\lstlistingname{Listagen}}
\addto{\captionsportuguese}{\renewcommand*\lstlistlistingname{Listagens}}
%\addto{\captionsportuguese}{\renewcommand*\acronymname{Acrónimos}}
\addto{\captionsportuguese}{\renewcommand{\bibname}{Refer\^{e}ncias}}
\addto{\captionsportuguese}{\renewcommand{\appendixname}{Anexo}}
\addto{\captionsportuguese}{\renewcommand{\appendixpagename}{Anexos}}
\addto{\captionsportuguese}{\renewcommand{\appendixtocname}{Anexos}}
%\noappendicestocpagenum
%%%%%%%%%%%%%%%%%%%%%%%%%%%%%%%%%%%%%%%%%%%%%%%%%%%%%%%%%%
\newtheorem{theorem}{Theorem}
\newtheorem{lemma}{Lemma}
\newtheorem{definition}{Defini\c{c}\~{a}o}
\newtheorem{notation}{Notation}
%%%%%%%%%%%%%%%%%%%%%%%%%%%%%%%%%%%%%%%%%%%%%%%%%%%%%%%%%%
\makeatletter
\renewcommand{\@makechapterhead}[1]{%
	\vspace*{50 pt}%
	{\setlength{\parindent}{0pt} \raggedright \normalfont
		\bfseries\Huge\thechapter.\ #1
		\par\nobreak\vspace{40 pt}}}
\makeatother
%%%%%%%%%%%%%%%%%%%%%%%%%%%%%%%%%%%%%%%%%%%%%%%%%%%%%%%%%%
%\renewcommand\thesection{\arabic{section}}
%\renewcommand\thesubsection{\thesection.\arabic{subsection}}
%%%%%%%%%%%%%%%%%%%%%%%%%%%%%%%%%%%%%%%%%%%%%%%%%%%%%%%%%%
\newcommand{\minipagespace}[1]{{\newline  \vspace{#1cm} \newline}}
\newcommand{\tablespace}[1]{{\vspace{#1cm}}}
\newcommand{\figurespace}[1]{{\vspace{#1cm}}}
\newcommand{\listingspace}[1]{{\vspace{#1cm}}}
%%%%%%%%%%%%%FIX SECTION NUMBERING IN CASE REPORT%%%%%%%%%
%\renewcommand\thesection{\arabic{section}}
%\renewcommand\thesubsection{\thesection.\arabic{subsection}}
%\renewcommand\thesubsubsection{\thesection.\thesubsection.\arabic{subsubsection}}
%%%%%%%%%%%%%%%%%%%%%%%%%%%%%%%%%%%%%%%%%%%%%%%%%%%%%%%%%%
\newcolumntype{L}[1]{>{\raggedright\arraybackslash}p{#1}}
\newcolumntype{C}[1]{>{\centering\arraybackslash}p{#1}}
\newcolumntype{R}[1]{>{\raggedleft\arraybackslash}p{#1}}
%%%%%%%%%%%%%%%%%%%%%%%%%%%%%%%%%%%%%%%%%%%%%%%%%%%%%%%%%%
\lstnewenvironment{java}[1][]{%
	\lstset{language=Java,#1}%
}{}
\newcommand*{\incjava}[1][]{%
	\lstinputlisting[{language=Java,#1}]%
}
%%%%%%%%%%%%%%%%%%%%%%%%%%%%%%%%%%%%%%%%%%%%%%%%%%%%%%%%%%
\newcommand\acrfullr[2][]{\acrshort[#1]{#2} (\acrlong[#1]{#2})}
%%%%%%%%%%%%%%%%%%%%%%%%%%%%%%%%%%%%%%%%%%%%%%%%%%%%%%%%%%
%%% to use only sections starting at ONE in BOOK mode
%\renewcommand\thesection{\arabic{section}}
%\renewcommand\thesubsection{\thesection.\arabic{subsection}}
%\renewcommand\thesubsubsection{\thesection.\thesubsection.\arabic{subsubsection}}
%%%%%%%%%%%%%%%%%%%%%%%%%%%%%%%%%%%%%%%%%%%%%%%%%%%%%%%%%%

