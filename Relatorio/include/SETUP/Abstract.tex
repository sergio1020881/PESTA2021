\begin{abstract}
\label{Resumo}
\addcontentsline{toc}{chapter}{RESUMO}
\par O projeto proposto é fazer uma balança utilizando um micro controlador, um sistema \textit{Embedded}.
\\
\\
Uma célula de peso vai ser o sensor de conversão entre massa e diferença de potencial através de uma ponte \textit{Wheatstone}, gerando um sinal proporcional.
\\
\\
Após obter este sinal será ligado a um amplificador \textbf{ADC} dedicado para este tipo de funcionalidade, com 24 bits de resolução, amplificação programável e taxa de transferência fisicamente programado, trata-se do integrado \textbf{HX711}, com um protocolo de comunicação que lhe é próprio. Depois esta comunicação serie vai ser entregue ao micro controlador.
\\
\\
A programação do \textbf{MCU}, o código as livrarias e ou drivers é para ser feito em linguagem \textbf{C}. O objetivo é para obter uma balança funcional de fácil utilização e calibração economicamente viável, assim  ficar com uma balança pratica.
%%A massa é uma das grandezas fundamentais [\textbf{LMT}] dai minha opção por apostar neste projeto devido a toda a física estar envolvida ao seu redor.
\vfill
\par 
\textbf{Palavras Chave:} \textit{Strain Gauge}, \textit{Load Cell}, Amplificador, Código, Programação, \textit{Embedded System} .
\end{abstract}
%%%%%%%%%%%%%%%%%%%%%%%%%%%%%%%%%%%%%%%%%%%%%%%%%%%%%%%%%%%%%%
