\begin{abstract}
\label{Resumo}
\par O projeto proposto passa por fazer uma balança, utilizando um micro controlador(\textit{Embedded System}). Uma célula carga será o sensor de conversão da diferença de potencial obtida de uma ponte \textit{Wheatstone} pela existencia de uma massa a pesar sobre a célula de carga, gerando um sinal proporcional ao seu peso.
\\
\\
Após obter este sinal, será ligado a um amplificador \ac{adc} dedicado para este tipo de funcionalidade, com 24 bits de resolução, com amplificação programável e com uma taxa de transferência fisicamente (\textit{hardware}) programada. Este amplificador é designado por \textbf{HX711}, e possui um protocolo de comunicação que lhe é próprio. Depois esta comunicação serie vai ser entregue a um \ac{mcu}.
\\
\\
A programação do \acrshort{mcu}, o código das livrarias e ou drivers será implementada em linguagem \textbf{C}. O objetivo é obter uma balança funcional de fácil utilização e calibração economicamente viável, desta forma uma balança acessível e prática.
%%A massa é uma das grandezas fundamentais [\textbf{LMT}] dai minha opção por apostar neste projeto devido a toda a física estar envolvida ao seu redor.
\\
\\
Este trabalho é orientado para quem tem alguns conhecimentos básicos de eletrotecnia, com orientação para a eletrónica e programação.
\vfill
\par 
\textbf{Palavras Chave:} \textit{Strain Gauge}, \textit{Load Cell}, Amplificador, Código, Programação, \textit{Embedded System} .
\end{abstract}
%%%%%%%%%%%%%%%%%%%%%%%%%%%%%%%%%%%%%%%%%%%%%%%%%%%%%%%%%%%%%%
\null
\setcounter{page}{0}
%%%%%%%%%%%%%%%%%%%%%%%%%%%%%%%%%%%%%%%%%%%%%%%%%%%%%%%%%%%%%%
\renewcommand{\abstractname}{Abstract}
\begin{abstract}
	\label{Summary}
	\par The proposed project is to build a wheighing scale using a microcontroller, an Embedded system.
	\\
	\\
	A load cell is what is going to be used to convert the measured wheight into a potencial difference with a wheatstone configuration, generating a proportional signal.
	\\
	\\
	After obtaing this signal it is going to be connected to a load cell amplifier with a \ac{adc} dedicated for this type of application, it has a 24 bit resolution, programable amplification and sampling rate fisically programmmable, this is the \textbf{HX711} chip, having a proprietary communication protocol.
	This information then will be passed to the microcontroller.
	\\
	\\
	The \ac{mcu} programming code, libraries and or drivers are to be implemented using the \textbf{C} language. The objective of this project is to get a functional wheighing scale easy to use and to calibrate economically viable, in order to have a practical wheight scale.
	\\
	\\
	This work is oriented to a public with some basic electrical knowledge with electronic and programming background.
	\vfill
	\par 
	\textbf{Keywords:} Strain Gauge, Load Cell, Amplifier, Code, Programming, Embedded System .
\end{abstract}