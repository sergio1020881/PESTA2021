\begin{abstract}
	\label{Resumo}
	\par O projeto proposto passa por fazer uma balança, utilizando um micro controlador (\textit{Embedded System}). Uma célula de carga será o sensor de conversão da diferença de potencial obtida por uma ponte \textit{Wheatstone} pela existência de uma massa a pesar sobre a célula de carga, gerando um sinal proporcional ao seu peso.
	\\
	\\
	Após obter este sinal, será ligado a um amplificador \ac{adc} dedicado para este tipo de funcionalidade, com 24 bits de resolução, com amplificação programável e com uma taxa de transferência fisicamente (\textit{hardware}) programada. Este amplificador é designado por \textbf{HX711}, e possui um protocolo de comunicação que lhe é próprio. Depois esta comunicação serie vai ser entregue a um \ac{mcu}.
	\\
	\\
	A programação do \acrshort{mcu}, o código das livrarias e ou drivers será implementada em linguagem \textbf{C}. O objetivo é obter uma balança funcional de fácil utilização e calibração economicamente viável, desta forma a obter uma balança acessível e prática.
	%%A massa é uma das grandezas fundamentais [\textbf{LMT}] dai minha opção por apostar neste projeto devido a toda a física estar envolvida ao seu redor.
	\\
	\\
	Este trabalho é orientado para quem tem alguns conhecimentos básicos de eletrotecnia, com orientação para a eletrónica e programação.
	%\vfill
	\vspace{5cm}
	\par 
	\textbf{Palavras Chave:} \textit{Strain Gauge}, \textit{Load Cell}, Amplificador, Código, Programação, \textit{Embedded System} .
\end{abstract}
