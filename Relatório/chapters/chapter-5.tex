\chapter{Validação do projeto implementado}
%%%%%%%%%%%%%%%%%%%%%%%%%%%%%%%%%%%%%%%%%%%%%%%%%%%%%%%%%%
%Neste capítulo deve apresentar os testes realizados ao sistema implementado, e que devem garantir os objetivos iniciais.
Neste capítulo são apresentados os testes realizados na implementação do projeto e explicado as características de funcionamento da balança.
%%%%%%%%%%%%%%%%%%%%%%%%%%%%%%%%%%%%%%%%%%%%%%%%%%%%%%%%%%
\section{Testagem e validação do sistema implementado}
%%%To validate is to justify why the choices made and alternatives that could be chosen.
O conhecimento adquirido ao aprofundar o funcionamento dos componentes é o ganho mais evidente na realização do projeto, que permitiu a interpretação de situações reais e deteção de anomalias (\textit{troubleshooting}).
\\
\\
Foram realizados testes, usando um osciloscópio Rigol DS1102E para afinações e resolução de anomalias e ajustando o código do anexo \ref{main-c}. O que deu mais trabalho foi a validação da biblioteca do amplificador de célula de carga HX711, que testada com os parâmetros permitidos e com diferentes frequências de transmissão. No código foram realizadas várias experiências para determinar o comportamento mais favorável. Na consulta do código que pode ser consultado nos anexos podem ver as estratégias tomadas e qual os procedimentos selecionados. Estes têm comentários com a possibilidade de alterar parâmetros que estão indicados.
%Esta justificação não é válida, tendo em conta o objetivo do trabalho !
%Apostei na marca \textbf{Atmel} devido a experiência e conhecimentos já adquiridos, se aposta-se noutra marca teria de enfrentar uma curva de aprendizagem e adaptação que no final a nível de custos beneficio seria desfavorável, pelo tempo a dispensar e de ser muito trabalhoso ao refazer tudo novamente noutra arquitetura.
\\
\\
\\
\\
O sensor usado é o mais comum neste tipo de aplicações pratica. Quanto ao circuito de interface a escolha foi baseada na sua precisão, ou seja, é de 24 \textit{bits} enquanto o \acs{adc} do \acs{mcu} é de apenas 10 \textit{bits} de resolução.
%Os 24 bits são necessarios? Os 10 não satifazem os objectivos ? Chega-se a este capítulo e isto não foi apresentado até agora
%%%%%%%%%%%%%%%%%%%%%%%%%%%%%%%%%%%%%%%%%%%%%%%%%%%%%%%%%%
\section{Algumas considerações acerca do funcionamento do sistema}
Quanto à funcionalidade no seu todo, a balança tem \textcolor{blue}{quatro} botões e \textcolor{blue}{três} \acsp{led}, um botão para fazer o \textit{offset} no \textcolor{green}{PORTF 0}, e dois botões com dupla função, fazer \textit{reset} para \textit{default} e incrementar, outro para entrar no menu de calibração e decrementar, o \textcolor{blue}{quarto} botão é reservado para \textit{enter} e assumir o valor introduzido na calibração.\\
\\
O botão \textcolor{green}{PORTF 3} quando premido durante \textcolor{blue}{cinco} segundos faz um \textit{reset} para configuração \textit{default} depois de o \acs{led} no \textcolor{red}{PORTC 6} piscar \textcolor{blue}{quatro} vezes.\\
\\
O botão \textcolor{green}{PORTF 4} quando premido durante \textcolor{blue}{cinco} segundos entra no menu de calibração do valor do \textit{gain factor} e o \acs{led} no \textcolor{red}{PORTC 7} liga. Usando os botões de incrementar e decrementar no
\textcolor{green}{PORTF 3} e \textcolor{green}{PORTF 4} pode-se alterar esse valor.\\
\\
Para assumir o valor e sair do menu de calibração basta premir o botão colocado no \textcolor{green}{PORTF 5}. Tanto no caso de calibração como no \textit{offset} os valores são guardados na \acs{eeprom} do micro-controlador, sendo que, se retirar a alimentação do circuito este não perde os valores e o \acs{led} no \textcolor{red}{PORTC 5} permanece ligado.
\\
\\
Nos ensaios foram utilizados objetos cujo peso era conhecido, não tendo pesos de calibração disponíveis para determinar sua precisão em toda a sua gama.
%%%%%%%%%%%%%%%%%%%%%%%%%%%%%%%%%%%%%%%%%%%%%%%%%%%%%%%%%%%%%%%%
\begin{comment}
Sem contar com as despesas no equipamento para a programação do hardware que em principio só se gasta uma vez, isto é, se não se estragar. No caso do programador \textbf{Atmel-ICE} pode custar até \EUR{185.55}.\\
\\
É de ter em conta que os preços são \textbf{PVP}, que no caso se for preços comerciais são dez vezes inferior, e se for para produção em grande escala também tem descontos por quantidade.\\
$\begin{array}{l l l}
\text{Média} & & \\
\overline{x} & = & \frac{1}{n}\sum_{i=1}^n x_i
\end{array}$
MEMS devices and structures are fabricated using conventional integrated circuit process techniques, such as lithography, deposition, and etching, together with a broad range of specially developed micromachining techniques. \cite{book-9}\\
The three essential elements in conventional silicon processing are deposition, lithography, and etching. \cite{book-9}\\
Sensitivity, Long-Term Drift and Temperature Effects (Span temperature hysteresis).
\end{comment}
%%%%%%%%%%%%%%%%%%%%%%%%%%%%%%%%%%%%%%%%%%%%%%%%%%%%%%%%%%%%%%%%
