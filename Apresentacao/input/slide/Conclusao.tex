\section{Conclusões}
\begin{frame}
	As conclusões que se pode tirar é a importância dos equipamentos ou ferramentas usadas no projeto tais como o multímetro e osciloscópio, que nos permite ter avanços significativos em afinações e ajustes. Além do já mencionado constantemente a necessidade de habilidade de interpretar \textit{datasheets} e manuais, um pre-requisito obrigatório que talvez nem é preciso o mencionar.
	\newline
	Acumular e documentar conhecimento é muito importante pois neste projeto recorri a livrarias que já tinha criado ha alguns anos e o estilo de programação seguindo uma metodologia sintática capaz de resolver qualquer problema com uma camada de abstração que simplifica significativamente o trabalho. 
	\newline
	\newline
	Considero que foi atingido os objetivos impostos, com possibilidade de no futuro melhorar, tais como funcionar por bateria e integrar um \textit{sleep mode} de forma a desligar o \textit{display} LCD e ficar em \textit{standby} até receber um sinal de \textit{wake up}.
	\newline
	\newline
	O projeto esta disponível no GITHUB link: \url{https://github.com/sergio1020881/PESTA2021/tree/main/SandBox/ATMEGA128/Atmega128}, e possível fazer download para quem quiser emular a experiência.
\end{frame}
